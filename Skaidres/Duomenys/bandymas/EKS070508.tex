%\documentclass[pdf,lt,slideColor,colorBG,noaccumulate,nototal,final]{prosper}
%% Norint matyti lietuviskas raides reikia susiinstaliuoti lietuvybes
%% arba unikodo palaikyma. Sitas dokumentas yra standartinėje ISO-8859-13 koduoteje. 
\documentclass[utf8,hyperref={unicode}]{beamer}
%\documentclass[utf8]{beamer}

\mode<presentation>
{
  \usetheme{Warsaw}
  % or ...
  % \usecolortheme{dove} 

  % or whatever (possibly just delete it)
   \setbeamertemplate{footline}
   {%
     \leavevmode%
    \hbox{\begin{beamercolorbox}[wd=.5\paperwidth,ht=2.5ex,dp=1.125ex,leftskip=.3cm 	plus1fill,rightskip=.3cm]{author in head/foot}%
    \usebeamerfont{author in head/foot}\insertshortauthor
    \end{beamercolorbox}%
    \begin{beamercolorbox}[wd=.5\paperwidth,ht=2.5ex,dp=1.125ex,leftskip=.3cm,rightskip=.3cm plus1fil]{title in head/foot}%
    \usebeamerfont{title in head/foot}\insertshorttitle \hfill p.
	\insertpagenumber\enspace iš \insertdocumentendpage\enspace
    \end{beamercolorbox}}%
  \vskip0pt%
}
%\hfill\insertshorttitle\enspace -- p. \insertpagenumber\enspace iš
%\insertdocumentendpage\enspace%
%   }
    
}
 \setbeamertemplate{navigation symbols}{}



\usepackage[english]{babel}
%\usepackage[utf8x]{inputenc}
\usepackage[L7x]{fontenc}
\usepackage{amsmath}
\usepackage{amssymb}
%\usepackage{theorem}
\usepackage{bm}
\usepackage{lmodern}
\usepackage{graphicx}

\newcommand{\ab}[1]{#1_{\alpha}}
\newcommand{\wab}[2][\delta]{w_{\alpha}(#2,#1)}
\newcommand{\normab}[1]{\lVert#1\rVert_{\alpha}}
\newcommand{\eps}{\varepsilon}
\newcommand{\sprod}[1]{\langle #1 \rangle}
\DeclareMathOperator{\diam}{diam}
%\renewcommand{\theenumi}{\roman{enumi}}
%\renewcommand{\labelenumi}{\theenumi)}
\theoremstyle{change}\newtheorem{teorema}{Teiginys}
\theoremstyle{change}\newtheorem{salyga}{}
%	\vspace*{20pt}
%	\vspace*{20pt}
\DeclareMathOperator{\seq}{seq}
\DeclareMathOperator{\Var}{Var}
\DeclareMathOperator{\tr}{tr}
\newcommand{\ds}[1]{\displaystyle{#1}}
\newcommand{\dlt}[2]{\Delta^{(#1)}_{#2}}
\newcommand{\norms}[1]{\lVert#1\rVert_{\alpha}^{\seq}}
\newcommand{\normh}[1]{\lVert#1\rVert}
\newcommand{\norma}[1]{\lVert #1\rVert_{\alpha}}

\newcommand{\skirt}[2]{\Delta^{(#1)}_{#2}}
\newcommand{\cp}{\buildrel P\over\longrightarrow}
\renewcommand{\theenumi}{\roman{enumi}}
\newcommand{\R}{\mathbb{R}}
%\newcommand{\E}{\mathbf{E}\,} % expectation operator
\newcommand{\bv}{\bm{v}}

\newcommand{\T}{T}
\newcommand{\n}{{\bm{n}}}
\newcommand{\jj}{{\bm{j}}}
\newcommand{\kk}{{\bm{k}}}
\newcommand{\bt}{\bm{t}}
\newcommand{\bu}{\bm{u}}
\newcommand{\B}{\bm{B}}
\newcommand{\N}{\mathbb{N}}
\newcommand{\bi}{\bm{i}}
\newcommand{\p}{\bm{\pi}}
\newcommand{\one}{{\bm{1}}}

\newcommand{\nni}{{\bm{n},\bm{i}}}
\newcommand{\nj}{{\bm{n},\bm{j}}}
\newcommand{\nk}{{\bm{n},\bm{k}}}
\newcommand{\kn}{\bm{k}_{\bm{n}}}

\newcommand{\vv}{\bm{\mathrm{v}}}
\newcommand{\vr}{{\mathrm{v}}}
\newcommand{\uu}{\bm{\mathrm{u}}}
\newcommand{\ur}{{\mathrm{u}}}


\newcommand{\abs}[1]{\left\vert #1 \right\vert}
\newcommand{\snk}{\sigma^2_{\bm{n},\bm{k}}}
\newcommand{\snj}{\sigma^2_{\bm{n},\bm{j}}}

\newcommand{\Rnj}{R_{\bm{n},\bm{j}}}
\newcommand{\Rnk}{R_{\bm{n},\bm{k}}}

\newcommand{\HH}{\mathrm{H}} % a set of notations for Holder spaces
\newcommand{\Ha}{\HH_{\alpha}}
\newcommand{\Hab}{\HH_{\alpha,\beta}}
\newcommand{\Habo}{\HH_{\alpha,\beta}^o}
\newcommand{\Hao}{\HH_{\alpha}^o}
\newcommand{\m}{\mathrm{m}}
\newcommand{\s}{\bm{s}}


\title[FCRT serijų schemai]{Funkcinė centrinė ribinė teorema serijų schemai}
%\shorttitle{FCRT daugiamačio indekso procesams}
\author[Zemlys]{Vaidotas Zemlys}
\institute[Vilnius University] {
    
    Vilniaus Universitetas
    \and
    
    Université des Sciences et Technologies de Lille
 }
\date{2007 gegužės 8 d.}

\begin{document}
\begin{frame}
    \titlepage
\end{frame}

\begin{frame}
    \frametitle{Sumavimo procesai} 
    \begin{itemize}
	\item Vienmatis
	    \begin{align*}
		\xi_n(t)=S_{[nt]}+(nt-[nt])X_{[nt]+1}
	    \end{align*}
	\item Gardelė:
	     \begin{align*}
	    R_{\bm{n},\bm{j}} := [(j_1-1)/n_1, j_1/n_1]\times\cdots\times
[(j_d-1)/n_d, j_d/n_d].
	    \end{align*}

	\item Daugiamatis:
	    \begin{align*}
		\xi_{\bm{n}}(\bm{t})=\sum_{\bm{1}\le \bm{j}\le
		\bm{n}}|R_{\bm{n},\bm{j}}|^{-1}|R_{\bm{n},\bm{j}}\cap
		[0,\bm{t}]|X_{\bm{j}}	
	    \end{align*}
    \end{itemize}
\end{frame}
\begin{frame}
    \frametitle{Sumavimo proceso konstrukcijos paaiškinimas} 
    \includegraphics{Rect.ps}
    \begin{align*}
	X=\frac{(1-a)(1-b)}{cd}A+\frac{a(1-b)}{cd}B+\frac{b(1-a)}{cd}D+
	\frac{ab}{cd}C
    \end{align*}
\end{frame}

\begin{frame}
    \frametitle{FCRT} 
	\begin{itemize}
	    \item Vienmačio argumento
		\begin{align*}
		    \frac{1}{\sqrt{n}}\xi_n\to W \text{ erdvėje } C([0,1])
		\end{align*}
	    \item Daugiamačio argumento
		\begin{align*}
		    \frac{1}{\sqrt{n_1\dots n_d}}\xi_{\bm{n}}\to W \text{
		    erdvėje } C([0,1]^d)
		\end{align*}
	\end{itemize}
\end{frame}
\begin{frame}
    \frametitle{Hiolderio erdvės} 
    \begin{itemize}
	   \item ${H}^o_\alpha\bigl([0,1]^d\bigr)$, $0<\alpha\le 1$ yra aibė tokių realių tolydžių
	funkcijų $x:[0,1]^d\to \mathbb{R}$, kurioms $\lim_{\delta\to   0}\wab{x}=0$, čia
	\begin{align*} %\label{f:walpha}
	    \wab{x}=\sup_{\bm{t},\bm{s}\in[0,1]^d,0<|\bm{t}-\bm{s}|<\delta}
	    \dfrac{|x(\bm{t})-x(\bm{s})|}{|\bm{t}-\bm{s}|^\alpha}.
        \end{align*}
	\item Separabili Banacho erdvė su norma
	    \begin{align*}
	    \normab{x}=|x(0)|+\wab[1]{x}.
	    \end{align*}
       \end{itemize}

\end{frame}
\begin{frame}
    \frametitle{Sąlygos} 
    \begin{itemize}
	\item $C([0,1]^d)$. Tokios pačios kaip ir CRT, $EX_{1}^2<\infty$.
	    Prokhorov (1956), Alexander ir Pyke (1986).
	\item $H_\alpha([0,1])$ :
	    \begin{align*}
		\lim_{t\to\infty}t^pP(|X_1|>t)=0
	    \end{align*}
	    Račkauskas ir Suquet (2001), čia $p=1/(1/2-\alpha)$.
	\item $H_\alpha([0,1]^d)$:
	    \begin{align*}
		\sup_{t>0}t^pP(|X_1|>t)<\infty
	    \end{align*}
	    Račkauskas, Suquet, Zemlys (2006).

    \end{itemize}
\end{frame}
\begin{frame}
    \frametitle{Serijų schema} 
    \begin{itemize}
	\item $\{X_{n,k},1\le k\le k_n, n\in \mathbb{N}\}$. Kiekvienam
	    $n$, $X_{n,k}$ nepriklausomi.
	\item Įprastinė konstrukcija „netinka“:
	    \begin{align*}
		E\xi_n(t)^2\approx\sum_{k=1}^{[nt]}\sigma^2_{n,k}
	    \end{align*}
	    čia $\sigma_{n,k}^2=EX_{n,k}^2$.
    \end{itemize}
\end{frame}
\begin{frame}
    \frametitle{Adaptyvi konstrukcija vienmačiu atveju} 
    \begin{itemize}
	\item $b_n(k)=\sum_{j=1}^{k}\sigma^2_{n,j}$. 
	\item Tegu $b_n(k_n)=1$, tada
	    \begin{align*}
		\xi_n(t)=S_n(k-1)+(t-b_n(k-1))\sigma^2_{n,k}X_{n,k},
	    \end{align*}
	kai  $b_n(k-1)< t\le b_n(k)$.

    \end{itemize}
\end{frame}
\begin{frame}
    \frametitle{Sąlygos} 
    \begin{itemize}
	\item $C([0,1])$, kaip ir CRT, 
	    \begin{align*}
		&\max_{1\le k\le k_n} \sigma^2_{n,k}\to 0\\
		&\sum_{k=1}^{k_n} EX_{n,k}^2I(|X_{n,k}|>\eps)\to 0
	    \end{align*}
	    Araujo ir Gine (1980)
	\item $H_{\alpha}^o([0,1])$, $q>p=1/(1/2-\alpha)$, 
	    \begin{align*}
		\sum_{k=1}^{k_n}\sigma_{n,k}^{-2q\alpha} EX_{n,k}^q\to 0,
	    \end{align*}
	    Račkauskas ir Suquet (2003)
    \end{itemize}
\end{frame}

\begin{frame}
    \frametitle{Netolygi gardelė daugiamačiu atveju} 
    \begin{itemize}
	\item $(X_{\bm{n},\bm{k}},\; \bm{1}\le \bm{j}\le \bm{k}_{\bm{n}}),\; \bm{n}\in
    \mathbb{N}^d$
\item 
    \begin{align*}
    S_\n(\bm{k})=\sum_{\bm{j}\le \bm{k}}X_\nj,\quad b_\n(\bm{k})=\sum_{\bm{j}\le
    \bm{k}}\snk
\end{align*}

	\item $ b_i(k)=b_\n(k^1_\n,\dots,k^{i-1}_\n,k,k^{i+1}_\n,\dots,k^d_\n)$
	\item Netolygi gardelė: 
	    \begin{align*}
    \Rnj :=\bigg[b_1(j_1-1),\; b_1(j_1)\bigg)\times \dots\times
    \bigg[b_d(j_d-1),\;b_d(j_d)\bigg)
\end{align*}
	 
    \end{itemize}
\end{frame}
\begin{frame}
    \frametitle{Rezultatas} 
    \begin{itemize}
	\item Sumavimo procesas
   \begin{align*}
		\xi_{\bm{n}}(\bm{t})=\sum_{\bm{1}\le \bm{j}\le
		\bm{n}}|R_{\bm{n},\bm{j}}|^{-1}|R_{\bm{n},\bm{j}}\cap
		[0,\bm{t}]|X_{\bm{j}}	
    \end{align*}
	\item Funkcinė centrinė ribinė teorema:
	    \begin{align*}
		\xi_\n\to W \text{ erdvėje } \ab{H}([0,1]^d),
	    \end{align*}
	    kai $\m(\bm{n})\to\infty$.
    \end{itemize}
 \end{frame}

\begin{frame}
    \frametitle{Sąlygos tirštumui} 
    \begin{itemize}
	\item Begalinio mažumo sąlyga:
	   \begin{align*}
		\max_{1\le l\le d}\max_{1\le k_l\le k^l_\n}\Delta b_l(k_l) \to 0,
	    \text{ kai } \m(\bm{n})\to \infty
	\end{align*}
    \item Momentinė sąlyga:
	
	\begin{align*}
    \lim_{\m(\n)\to\infty}
    \sum_{\one\le \bm{k}\le\kn}\sigma_{\nk}^{-2q\alpha}E|X_\nk|^q=0,
\end{align*}
    čia $q>p:=1/(1/2-\alpha)$.
    \end{itemize}
 
\end{frame}

\begin{frame}
    \frametitle{Apibrėžimai} 
    Kiekvienai funkcijai $x\in {H}^o_\alpha([0,1]^d)$:
    \begin{align*}
	\lambda_{0,\bm{v}}(x)&=x(\bm{v}),\quad \bm{v}\in V_0,\notag \\
	\lambda_{j,\bm{v}}(x)&=x(\bm{v})-\dfrac{1}{2}(x(\bm{v}^-)+x(\bm{v}^+)),\quad
	\bm{v}\in V_j,\ \ j\ge 1,
    \end{align*}
    čia $V_j=W_j\backslash W_{j-1}$, $W_j=\{k2^{-j};0\le k\le 2^j\}^d$.
\begin{align*}
  v_i^\pm=
  \begin{cases}
    v_i\pm2^j, & \text{kai}\; k_i\; \text{yra nelyginis};  \\
     v_i,& \text{kai}\; k_i\; \text{yra lyginis}.
  \end{cases}
\end{align*}    
\end{frame}
\begin{frame}
    \frametitle{Tirštumo kriterijus} 
   \begin{teorema}\label{t:tight}
    Tegu $\{\phi_{\bm{n}}, \bm{n}\in\mathbb{N}^d\}$ ir $\phi$ yra atsitiktiniai
    elementai su reikšmėmis erdvėje $H^o_\alpha([0,1]^d).$ Tarkime yra tenkinamos
    sąlygos.
\begin{itemize}
    \item[i)] Kiekvienam diadiniam $\bm{t}\in [0,1]^d$, 
	 apibendrinta seka $\phi_{\bm{n}}(t)$ yra
	asimptotiškai tiršta.
   \item[ii)]  Kiekvienam $\eps>0$
    \begin{align*}
        \lim_{J\to \infty} \limsup_{\bm{n}\to\infty}P(\sup_{j\ge J}2^{\alpha j}\max_{\bm{v}\in
        V_j}|\lambda_{j,\bm{v}}(\phi_{\bm{n}})|>\eps) = 0.
    \end{align*}
\end{itemize}
Tada apibendrinta seka $\phi_{\bm{n}}$ yra asimptotiškai tiršta erdvėje
$\ab{H}^o([0,1]^d)$.
\end{teorema}
\end{frame}


\begin{frame}
    \frametitle{Prielaidos baigtiniamačių pasiskirstymų konvergavimui} 
    \begin{itemize}
	\item $\mathcal{J}$ visos baigtinės aibių $\{[0,\bm{t}]\}$
	    sankirtos.
	    
	\item ``Dispersijos matas'':
	  \begin{align*}
	     \mu_\n(C)=\sum_{\bm{1}\le\bm{k}\le\kn}\one\{\B(\bm{k})\in C\} \snk
	    \end{align*}
	\item Prielaida:	kiekvienam  $C\in \mathcal{J}$ 
	    \begin{align*}
		\lim_{\m(\bm{n})\to\infty}\mu_\n(C)\to |C|.
	    \end{align*}
  
    \end{itemize}
   \end{frame}
\begin{frame}
    \frametitle{Baigtiniamačių skirstinių problema} 
    \begin{itemize}
	\item Serijų schema: $\kn=(2n,2n)$
     \begin{align*}
	\sigma_{\n,\bm{k}}&=\frac{1}{10n^2}, \bm{k}\le (n,n)\\
	\sigma_{\n,\bm{k}}&=\frac{3}{10n^2}, \bm{k}\in
	\{1,\dots,2n\}^2\backslash\{1,\dots,n\}^2\\
    \end{align*}
    \item $E\xi_\n^2(1/2,1/2)=1/10$, kai $EW^2(1/2,1/2)=1/4$.
\end{itemize}
\end{frame}


\begin{frame}
    \frametitle{Goldie ir Greenwood (1986) rezultatas}
	\begin{itemize}
	    \item Naudojama tolygi gardelė. Reikalaujama baigtiniamačių
		skirstinių konvergavimo.
	    \item FCRT tolydžių funkcijų erdvėje.  
	    \item Momentinė sąlyga: seka $(n^{d/2}X_{n,k})^s$ turi būti tolygiai
		integruojama, kuriam nors $s>2$.
	\end{itemize}
\end{frame}
\begin{frame}
    \frametitle{Bickel ir Wichura (1971) rezultatas} 
    \begin{itemize}
	\item Atvejis $d=2$, $EX^2_{\n,(i,j)}=a_ib_j$.
	\item \begin{align*}
		S_n(t_1,t_2)=\sum_{i\le A_n(t_1)}\sum_{j\le
		B_n(t_2)}X_{n,(i,j)}.
	    \end{align*}
	\item Jei tenkinama Lindebergo ir begalinio mažumo sąlyga,
	    $S_n(\bm{t})$ konverguoja į Vynerio paklodę erdvėje $D([0,1]^2)$.		          \end{itemize}
\end{frame}
\begin{frame}
    \frametitle{Tolimesni planai} 
    \begin{itemize}
	\item Sąlygos tolygiai gardelei.
	\item Momentinių sąlygų susilpninimas.
	\item Alternatyvi gardelės konstrukcija.
    \end{itemize}
\end{frame}

\end{document}

