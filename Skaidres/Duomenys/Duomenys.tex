\documentclass[12pt,a4paper]{article}
\usepackage[utf8]{inputenc}
\usepackage[L7x]{fontenc}
\usepackage[left=3.25cm,right=3.25cm]{geometry}
\usepackage[lithuanian]{babel}
\usepackage{lmodern}
\usepackage{graphicx}
\usepackage{url}
\usepackage{hyperref}
\usepackage{amssymb,amsmath}
\usepackage{theorem}
\usepackage{calc}
\usepackage{color}
\usepackage{bm}
\usepackage{verbatim}
\usepackage{soul}
\usepackage{hyperref} 
\usepackage{multicol}
\usepackage{indentfirst}
\usepackage{multirow}
\usepackage{tabularx}
\usepackage{makeidx}
\usepackage{float}
\usepackage[toc,page]{appendix}
%\usepackage{tocbibind}

\oddsidemargin=0cm
%\topmargin=1cm
\headsep=0pt
\textwidth 6.5in
\textheight 9.00in
%\headheight=0pt
%\textwidth=440pt
%\textheight=640pt
%\footskip=40pt
\makeatletter
\renewcommand\paragraph{%
   \@startsection{paragraph}{4}{0mm}%
      {-\baselineskip}%
      {.5\baselineskip}%
      {\normalfont\normalsize\bfseries}}
\makeatother

\begin{document}
	\begin{center}{\large\textbf{HLM modeliai TIMSS duomenims}}\end{center}

\section{Turinys}
Šiandien kuo trumpiau pasistengsiu papasakoti apie savo turimus duomenis ir šiek tiek iš praėjusios paskaitos skolų :)
\section{Įvadas}
Tyrimas TIMSS (Trends in International Mathematics and Science Study) – tai tarptautinis matematikos ir gamtos mokslų gebėjimų tyrimas, kas ketverius metus vykdomas daugelyje šalių, esančių beveik visuose pasaulio žemynuose.

Pagrindiniai klausimai: kas, ko ir kaip yra mokoma klasėse, ką mokiniai išties išmoksta ir kaip jie įgytas žinias vertina?

Tyrimo TIMSS tiriamasis objektas – ketvirtos ir aštuntos klasės mokinių matematikos ir gamtos mokslų pasiekimai bei jų kaita. Pasiekimai suskirstyti pagal įvairius kriterijus: turinio ir gebėjimų sritis, skirtingus pasiekimų lygmenis. Be pasiekimų, tyrime TIMSS daug dėmesio skiriama informacijai apie ugdymo kontekstą: mokyklos išteklius, ugdymo programų ir paties ugdymo kokybę. Nagrinėjant kontekstą, taip pat renkama informacija apie mokinio namų aplinką, šeimossocialinę ekonominę padėtį. 

Tarptautinis matematikos ir gamtos mokslų tyrimas TIMSS (Trends in International Mathematics and Science Study) inicijuotas tarptautinės švietimo pasiekimų vertinimo asociacijos IEA. IEA yra nepriklausoma tarptautinė organizacija, kurią sudaro įvairių  šalių valstybinės švietimo institucijos bei tyrimų centrai. Pagrindinis šios organizacijos, įkurtos 1958 m., tikslas – vykdyti lyginamuosius švietimo politikos, mokymo( si) praktikos ir mokinių pasiekimų tyrimus. Dalyvauja apie 70 šalių iš viso pasaulio. Pagrindinė IEA būstinė yra Amsterdame, Olandijoje.


\section{TIMSS klausimynų ir imčių sudarymas}
\subsection{Šalių skirtumai}
TIMSS tyrime sudarant klausimynus bei uždavinius didelis dėmesys skiriamas šalių skirtumams bei suvienodinimui. Kiekviena šalis gauna klausimyną, kuriame užpildo savo duomenis, bei aprašo mokymo programas. Tuomet suvažiuoja visi atstovai ir aptaria, kas keikvienai šaliai yra svarbu ir nusprendžia, kurias bendras sritis tirs. Stengiamasi suderinti standartus kiekviename žingsnyje, taip yra užtikrindamas tyrimo patikimumas.
\subsection{Tyrimo medžiaga}
\begin{itemize}
\item Testo sąsiuviniai - sudaromi 28 testo sąsiuviniai (po 14 4 ir 8 klasėm). Vienam mokiniui atitenka vienas sąsiuvinys su 4 užduočių blokų (2 matematikos ir 2 gamtos mokslų). Sąsiuviniai išdalinami mokyniams iš anksto numatyta tvartka, o ne atsitiktinai. Pusė atvirų klausimų ir kita pusė pasirenkamo atsakymo.
\item Klausimynas apie mokymo programas
\item Mokinio klausimynas - tiria mokinio mokyklos bei šeimos gyvenimo aspektus: demografinę informaciją, namų aplinką,
mokyklos atmosferą, mokinio savivoką bei nuostatas, susijusias su matematikos ir gamtos mokslų mokymusi.
\item Mokytojo klausimynas - mokytojų charakteristikas, klasės kontekstą ir dėstomas temas surinkti. jų išsimokslinimą, profesinį tobulėjimą, požiūrį į galimybes bendradarbiauti su kitais mokytojais, pasitenkinimą darbu irpan. klasių, dalyvaujančių 2011 m. tyrime TIMSS, charakteristikas, mokymo laiką, mokymui naudojamą medžiagą bei veiklas matematikos ir gamtos mokslų pamokose.
\item Mokyklos klausimynas - mokyklos charakteristikas, mokymo laiką, išteklius ir technologijas, tėvų dalyvavimą mokyklos gyvenime, mokyklos atmosferą, mokytojų personalą, mokyklos direktoriaus vaidmenį ir bendrą mokinių pasirengimą.
\end{itemize}
\subsection{Imtys}
TIMSS imtis yra dviejų pakopų sluoksninė lizdinė, kurios pirmos pakopos elementai išrenkami su tikimybėmis, proporcingomis dydžiui, antos pakopos elementai išrenkami su lygiomis tikimybėmis. Pirmoje pakopoje pagal aštuntokų skaičių mokykloje. O antoje atrenkamos klasės atsitiktinai. Kiekvienai atrinktai mokyklai parenkama po dvi dubleres su panašiom charakteristikom, kad atsisakius, būtų galima pakeisti kita. Tyrime privalo dalyvauti nemažiau kaip 80\% atrinktų mokinių, kad būtų išvengta tyčinio prastesnių mokinių nedalyvavimo.
\section{Klausimų pavyzdys}
\section{Duomenų apdorojimas, imtys}
Kadangi tiriama ne visa apopuliacija, o jos atstovai. Tyrimo rezultatai primetami visai populiacijai. Tai padaroma naudojant svorius pavaizduotus skaidrėje. Tokie svoriai sukuriami ir mokykloms.
\section{Duomenų apdorojimas, plausible values}
Kadangi testų atlikimo laikas ribotas ir ne visi mokiniai spėja atsakyti į visus klausimus ir norima suvienodinti taptautinius rezultatus, tikrasis testo rezultatas yra perskaičiuojamas į dydį, kurio vidurkis per visas šalis yra 500, o standartinis nuokrypis 100. Tokių kintamųjų sudaromi 5. Jie vadinami plausable values.

Rezultatai sudaromi pasitelkiant moderniosios testų teorijos metodologiją (\textit{angl. IRT}), kuri remiasi psichometriniais modeliais. Sodaromas skirstinys kiekvienam mokiniui ir iš jo sugeneruojamos 5 reikšmės.



\end{document}

	

	
