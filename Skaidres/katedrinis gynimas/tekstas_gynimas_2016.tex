\documentclass[12pt,a4paper]{article}
\usepackage[utf8]{inputenc}
\usepackage[L7x]{fontenc}
\usepackage[left=3.25cm,right=3.25cm]{geometry}
\usepackage[lithuanian]{babel}
\usepackage{lmodern}
\usepackage{graphicx}
\usepackage{url}
\usepackage{hyperref}
\usepackage{amssymb,amsmath}
\usepackage{theorem}
\usepackage{calc}
\usepackage{color}
\usepackage{bm}
\usepackage{verbatim}
\usepackage{soul}
\usepackage{hyperref} 
\usepackage{multicol}
\usepackage{indentfirst}
\usepackage{multirow}
\usepackage{tabularx}
\usepackage{makeidx}
\usepackage{float}
\usepackage[toc,page]{appendix}
%\usepackage{tocbibind}

\oddsidemargin=0cm
%\topmargin=1cm
\headsep=0pt
\textwidth 6.5in
\textheight 9.00in
%\headheight=0pt
%\textwidth=440pt
%\textheight=640pt
%\footskip=40pt
\makeatletter
\renewcommand\paragraph{%
   \@startsection{paragraph}{4}{0mm}%
      {-\baselineskip}%
      {.5\baselineskip}%
      {\normalfont\normalsize\bfseries}}
\makeatother

\begin{document}
	\begin{center}{\large\textbf{Dviejų lygių HLM modelių vertinimo metodų palyginimas MC simuliacijų būdu: REML vs MINQUE}}\end{center}

\section*{Įžanga}

Labas rytas. Aš Eglė Kaleckaitė šiandien pristatysiu savo darbą : dviejų ... darbo vadovas -- Vydas Čekanavičius.


\section*{Turinys???ar reikia iš viso}
Apie ką kalbėsiu:
\begin{itemize}
\item pristatysiu savo darbo tikslus ir problemas;
\item peržvelgsiu, kas yra HLM ir TIMSS;
\item kaip atrodo vertinimo metodai;
\item peržiūrėsim kokios simuliacijos atliktos ir kokie gauti rezultatai;
\item galiausiai pristatysiu sudarytą TIMSS modelį ir apibendrinimas.
\end{itemize}


\section*{Tikslai ir Problemos}
 Dažniausiai HLM yra vertinami REML procedūra, darant paklaidų ir atsitiktinių efektų normalumo prielaidą. Tačiau yra įrodyta, jog REML dispersijos komponenčių įverčiai yra paslinkti į mažesniąją pusę, todėl ir fiksuotų efektų standartinės paklaidos gaunamos mažesnės nei iš tikrųjų, o to pasekoje atmetama hipotezė, jog kintamasis yra statištiškai reikšmingas, nors jis yra reikšmingas. Todėl buvo tirtas alternatyvus dispersijos komponenių vertinimo metodas MINQUE, kuris nereikalauja jokių žinių apie skirstinį. Tad šio darbo tikslas yra palyginti du dispersijos komponenčių vertinimo metodus dviejų lygių hierarchiniams tiesiniams modeliams. Tikėtasi, jog alternatyvus metodas duos mažiau paslinktus dispersijos komponenčių įverčius. Labiausiai orientuojamasi į TIMSS (tiksliau 8 klasių matematikos testo rezultatus), kurį tuoj pristatysiu, į šio tipo duomenų struktūrą. Norėta vertinti su laisvai prieinamu atviro kodo statistinės analizės paketu R. REML vertinimui pavyko rasti paketą ir funkcijas. MINQUE nepavyko, todėl buvo parašytas R funkcijų rinkinys. Dar susidurta su dviem problemomis. TIMSS duomenys atkeliauja su imties svoriais į kuriuos reikėtų atsižvelgti vertinant, ir atkeliauja 5 galimos testo reikšmės. Į antąją problemą šiame darbe atsižvelgta nebuvo, galima tiesiog gauti šverčius kiekvienai reikšmei ir suvidurkinti. O R vertinimui su imties svoriais fukcijos nepavyko rasti. Todėl buvo pritaikytas PWIGLS svėrimo metodas MINQUE metodui ir simuliacijomis patikrinta, jog poslinkis gaunamas mažesnis, kai turime informatyvų ėmimą. Galiausiai sudarytas modelis su atsitiktiniu postūmiu ir posvyriu Lietuvos aštuntokų TIMSS 2011 duomenims ir įvertintas minėtais metodais.


\section*{HLM}
HLM - hierarchinis tiesinis modelis (\textit{angl. Hierarchical Linear Model}). Skaidrėje, kur antras taškiukas, pateikta dviejų lygių HLM išraiška. Pirmoji lygtis yra kaip standartinė tiesinio modelio, bet betos yra taip pat dar priklausomos nuo kažkokių kintamųjų, kurie suskaidyti į grupes. Tokį modelį galima suvesti į jungtinį pavidalą, kuris dar vadinamas mišrių efektų tiesiniu modeliu. Arba galima suvesti į dar bendresnį matricinį pavidalą. Tipinis duomenų, kuriems taikomas HLM pavyzdys, yra mokyklų duomenys, tokie kaip TIMSS, kur pirmame lygmenyje yra mokiniai, kurie suskaidyti į klases, o klasės yra mokyklose. Galime plėstis dar, mokyklos yra miestuose, o miestai šalyse ir t.t.

\section*{TIMSS}
Tokie yra TIMSS duomenys. Tai yra tarptautinis matematikos ir gamtos mokslų tyrimas. Jis yra vykdomas kas 4 metus visuose žemynuose, kažkur apie 70 šalių. Sudaroma dviejų pakopų sluoksninė lizdinė imtis pagal kiekvinos šalioes poreikius ir TIMSS standartus. Pirmoje pakopoje parenkamos mokyklos iš sudarytų lizdų su tikimybėmis proporcingomis mokyklos aštuntokų skaičiui. Antoje pakopoje parenkamos klasės su vienodomis tikimybėmis ir apklausiami visi parinktos klasės mokiniai, jų tėvai, mokytojai bei užpildomas mokyklai skirtas klausimynas. Galiausiai duomenys apdorojami taip, jog bendras visų tyrime dalyvavusių mokinių testo rezultaų vidurkis būtų 500, o standartinis nuokrypis 100. Sudaromos 5 galimos rezultato reikšmės kiekvienam mokiniui. Tokių duomenų aiškinantieji kintamieji mokyklos, mokinio ir klasės bei mokytojo lygiams yra indeksai (pvz.: mokyklos mokytojų nuomone mokykla kreipia dėmesį į mokyklos akademinę sėkmę (sudarytas iį kelių kategorinių kintamųjų)), žymimieji kintamieji (pvz.: mokinio lytis) ir dydžio kintamieji (mokyklos dydis, klasės dydis).

\section*{REML}
Jau nupasakojau, kas yra hierarchiniai tiesiniai modeliai, kur jie taikomi. Dabar papasakosiu apie dažniausiai taikomą metodą REML. Tai yra apribotasis didžiausio tikėtinumo metodas, kurį pasiūlė Harley ir Rao. Mes galime aksčiau minėtą mišrių efektų matricinę išraišką suskaityti pagal antro lygio subjektus, kaip parodyta skaidrėje. Dažniausiai yra daroma prielaida, jog tokio modelio paklaidos $\varepsilon$ ir atsitiktiniai efektai u yra psiskirstę pagal normalųjį dėsnį. Tuomet REML dispersijos komponenčių įverčiai yra gaunami maksimizuojant log-tikėtinumo funkciją, pateiktą skaidrėje. O fiksuoti parametrai tuomet gaunami standartiškai pagal GLS.

\subsection*{REML privalumai trūkumai}
Tikriausiai būtų galima ir daugiau jų rasti, bet čia keli. Šio metodo privalumas tas jog jis paprastas, greitai suskaičiuojamas,l turi mažesnį poslinkį nei ML. Bet visgi taip vertinant daroma paklaidų ir atsitiktinių efektų normalumo prielaida, negarantuojamas įverčių nepaslinktumas ir neneigiamumas, o procedūra yra iteratyvi.

\section*{MINQUE}
Todėl šiame darbe, kaip jau minėjau, tiriamas alternatyvus metodas -- MINQUE -- mažiausios normos kvadratinis nepaslinktas įvertinys. Sakykim, kad turime mišrių efektų modelį kaip anksčiau išreikštą benta matricine forma. Jei mes dispersiją galime išskaidyti tiesine nežinomų parametrų ir žinomų matricų kombinacija, tuomet tiesinės  kombinacijos g1... mažiausios normos kvadratinis nepaslinktas įvertinys yra  Y'AY, kur matrica A gaunama minimizatus tr(AVAV) su invariantiškumo ir nepaslinktumo sąlygomis. Tuomet dispersijos komponenčių įverčiai gaunami pagal žemiau pateiktas formules. O fiksuotų parametrų įverčiai gaunami standartiškai.


\section*{MINQUE privalumai ir trūkumai}
Šis įvertinys pasižymi trokštamomis savybėmis kaip nepaslinktumas ir invariantiškumas. Be to nereikia žinoti paklaidų ir atsitiktinių efektų skirstynio. Tačiau trūkumas tas, jog šis įvertinys yra nepaslinktas tik prie turimų a priori reikšmių ir 100 žmonių su tais pačiais duomenimis gali gauti šimtą skirtingų įverčių ir jie visi bus nepaslinkti. Šis metodas taip pat negarantuoja neneigiamų disversijos komponenčių įverčių, o vertinimas reikalauja didelių matricų apvertimo ir užima daug laiko bei reikalauja daug kompiuterio resursų.

\section*{A priori}
Dar šiek tiek apie apriori reikšmes kurios buvo naudotos šiame darbe. Tai tegul a.... žymi apriori reikšmes. Tai kai pirmoji reikšmė, kuri yra prie sigmos prilyginama 1, o visos kitos 0, tuomet turime MINQUE(0). Kai visos reikšmės prilyginamos 1, turime MINQUE(1). Jei apriori reikšmes galime nuspėti iš kitų tyrimų arba gaunami kitais metodais, tuomet turime MINQUE($\theta$). Pavadinkime MINQUE su skirtingomis apriori reikšmėmis, tiesiog skirtingais metodais. Šiame darbe naudojami visi minėti metodai. MINQUE($\theta$) apriori reikšmės gaunamos iš dviejų žingsnių mažiausių kvadratų metodo. Kuomet kiekvienam antro lygio subjektui vertinamas modelis ... ir gaunami beta įverčiai. Tuomet vertinami modeliai betoms. Taip gaunami visi reikiami įverčiai. A priori reikšmėms naudojami ne patys įverčiai. O dalinama iš sigmos įverčio visi ir taip gaunami santykiai.

Praktikoje dažnai naudojamas I--MINQUE, kuris nepriklauso nuo apriori reikšmių. Čia ankstesnės iteracijos įverčiai naudojami kaip \textit{a priori} reikšmės kitoje iteracijoje, pradinės reikšmės imamos vienetai. Šiuo metodu gauti įverčiai yra tapatūs REML įverčiui. O MINQUE su vienodomis pradinėmis reikšmėmis kaip REML yra pirmoji REML iteracija, todėl šis metodas praktikoje dažnai naudojamas REML pradinėms reikšmėms gauti.


\section*{Buvę tyrimai}
Tyrimų buvo atlikta daug ir įvairių, bet ne hierarchinėms struktūroms. Tai paminėsiu tik du dėl kurių ir pasirinkau tokią temą. Bagaka's  suvedė MINQUE($\theta$) metoda dviejų lygių HLM su atsitiktiniu postūmiu. Šiam metodui pritaikė saviranką ir palygino abu metodus, kai paklaidos ir atsitiktiniai efektai pasiskirstę pagal normalųjį ir dvigubą eksponentinį dėsnius. Tirta įvairioms reikšmėms ir dydžiams. Dispersijos komponentėms gauti įverčiai labai panašūs abiems metodams.

Delpish savo disertacijoje panaudojo Bagaka's idėją, tačiau lygino REML su dviejų iteracijų MINQUE su saviranka modeliui su atsitiktiniu postūmiu ir posvyriu. Tirti taip pat keli variantai bei normaliosios ir chi paklaidos. Gauti rezultatai parodė, jog fiksuotų parametrų įverčiai gauti tikslūs, nors MINQUE su saviranka poslinkis mažesnis, tačiau MINQUE su saviranka gauti pasiklaiutiniai intervalai tikslesni.


\section*{Šio darbo simuliacijų struktūra}
Šiame darbe stengtasi atsižvelgti į kintamųjų sudarymą ir imčių stuktūrą. Sudaromas modelis su atsitiktiniu postūmiu ir posvyriu kaip skaidrėje. Pirmo lygio paklaidos ir atsitiktiniai efektai generuoti pasiskirstę pagal normalųjį ir chi desnius. Chi standartizuotas ir paskui padauginta iš standartinio nuokrypio. Nagrinėjami atvejai bus kitoje skaidrėje. Pašneku apie lenteles..

\section*{Simuliacijų rezultatai}

\begin{itemize}
\item Visų fiksuotų parametrų ir $\sigma^2$ įverčių santykinis poslinkis mažesnis nei 5\%;
\item Poslinkis didesnis ir mažesnis tikslumas, kai turimos $\chi^2$ paklaidos visiems metodams ir nagrinėtiems atvejams;
\item Didžiausias santykinis poslinkis dispersijos komponentėms gautas kuomet $\frac{\tau_{00}}{\sigma^2}=0,05$, nesvarbu koks paklaidų pasiskirstymas ar antro lygio subjektų skaičius;
\item MINQUE(1) anksčiau minėtu atveju poslinkis mažesnis nei REML, o įverčių išsibarstymas nedaug didesnis;
\item Beveik visais atvejais MINQUE($\theta$) įverčių poslinkis  ir išsibrstymas mažiausi, kai dizainas nesubalansuotas;
\item Kuo didesnis antro lygio objektų skaičius tuo mažesni įverčių vidutiniai santykiniai poslinkiai bei išsibarstymai ir skirtumas tarp REML ir MINQUE(1), MINQUE($\theta$).
\end{itemize}

\section*{PWMINQUE}
Kaip jau minėjau, TIMSS duomenys pateikiami su imties svoriais, tad buvo sukurtas PWMINQUE metodas, kuris pagrįstas PWIGLS svėrimo metodu. Čia nesiplėsiu, viskas aprašyta darbe. gana sunku buvo sukurti simuliacijų struktūrą, tai buvo pasirinkta tokia pati kaip naudota Steele ir kt. Nesiplėsiu. Modelis įvertintas metodais ant skaidrės. Gauti rezultatai parodė, jog svertiniais metodais gauti įverčiai turėjo žymiai mažesnį poslinkį nei nesvertiniai.

\section*{TIMSS modelis}
Galiausiai sudarytas modelis minėtiems tims duomenims. Šis modelis įvertintas visais metodais išskyrus MINQUE(0). MINQUE(1) metodu įvertintos atsitiktinių efektų kovariacijos matricos gautos neigiamai aprėžtos. Taip nutiko dėl to jog įverčių dydžiai labai skirtingi, o jiems priskiriamas vienodas svoris. Kaip matome, REML ir MINQUE($\theta$) šverčiai ko ne identiški, o PWMINQUE($\theta$) kažkiek skiriasi.


\section*{Apibendrinimas}
\indent Pagrindiniai šio magistro darbo rezultatai:
\begin{itemize}
\item Parašytos R funkcijos MINQUE metodui, kuris skirtas HLM modelių parametrų vertinimui.
\item Sukurtas PWMINQUE metodas, simuliacijų būdu parodyta, jog šis metodas duoda mažiau paslinktus įverčius nei nesvertiniai metodai.
\item Atliktos empirinės simuliacijos, kurios parodė, jog išskyrus vieną atvejį ($\frac{\tau_{00}}{\sigma^2}=0,05$) REML, MINQUE(1) ir MINQUE($\theta$) metodai yra panašūs. O tai nepateisino lūkesčių, jog pažeidus normalumo sąlyga MINQUE metodu gauti įverčiai bus mažiau paslinkti.
\item Sudarytas hierarchinis tiesinis modelis su atsitiktiniu postūmiu Lietuvos matematinio raštingumo testo rezultatams, šis modelis įvertintas REML, MINQUE(1), PWMINQUE(1), MINQUE($\theta$) ir PWMINQUE($\theta$).
\end{itemize}

\indent Tolimesnis tyrimas galėtų apimti daugiau metodų fiksuotiems ir atsitiktiniams efektams vertinti (pvz.: Hamilton'o III metodas, AUE ir pan.), palyginti metodus su saviranka, taikyti sumuštinio (\textit{angl. sandwitch}) principo ar kt. standartines paklaidas, įtraukti daugiau dispersijos komponenčių reikšmių, pabandyti kitus paklaidų pasiskirstymus. Sukurti paketą R aplinkoje MINQUE vertinimui. MINQUE metodui pritaikyti kitų autorių sukurtus be matricinius pavidalus ir juos testuoti.


\end{document}

	

	
