\documentclass[utf8,hyperref={unicode,pdftex}]{beamer}
%\documentclass[utf8]{beamer}

\mode<presentation>
{
\usetheme{Warsaw}
}
\setbeamertemplate{navigation symbols}{}

\usepackage[utf8]{inputenc}
\usepackage[L7x]{fontenc}
\usepackage[lithuanian]{babel}
\usepackage{lmodern}
\usepackage{amsmath}
\usepackage{amssymb}
\usepackage{bm}
\usepackage{graphicx}
\usepackage{multirow}
\usepackage{subcaption}



\beamersetuncovermixins{\opaqueness<1>{25}}{\opaqueness<2->{15}}

\title[\hspace{130pt} p. \insertpagenumber\enspace iš \insertdocumentendpage\enspace ]{Dviejų lygių HLM modelių vertinimo metodų palyginimas MC simuliacijų būdu: REML vs MINQUE}
\author[ E. Kaleckaitė]{Eglė Kaleckaitė\\
Vadovas: Vydas Čekanavičius}
\institute{Vilniaus Universitetas, Matematikos ir Informatikos Fakultetas}
\date{2016 sausio 14d.}
\begin{document}
\begin{frame}
\titlepage
\end{frame}
%%%%%%%%%%%%%%%%%%%%%%%%%%%%%%%%%%%%%%%%%%%%%
%%%%%%%%%%%%%%%%%%%%%%%%%%%%%%%%%%%%%%%%%%%%%
\begin{frame}
\frametitle{Įvadas}
\begin{itemize}
\item Tikslai ir problemos;
\item HLM ir TIMSS;
\item Vertinimo procedūros: REML ir MINQUE;
\item Atliktos simuliacijos ir rezultatai;
\item TIMSS modelis;
\item Apibendrinimas.
\end{itemize}
\end{frame}

%%%%%%%%%%%%%%%%%%%%%%%%%%%%%%%%%%%%%%%%%%%%%
%%%%%%%%%%%%%%%%%%%%%%%%%%%%%%%%%%%%%%%%%%%%%
\begin{frame}
\frametitle{Tikslai ir problemos}
\begin{itemize}
\item Orientacija į TIMSS (\textit{angl. Trends in International Mathematics and Science Study}) ir į R;
\item HLM modeliams vertinami REML metodu, kuris remiasi normalumo prielaida;
\item Hierarchinėms struktūroms REML įverčiai yra paslinkti, mažesni nei iš tikrųjų (Timm\footnote{Timm, N.H., \textit{Applied Multivariate Analysis}, Springer Texts in Statistics, 2002});
\item Tiriamas alternatyvus metodas MINQUE, kuris nereikalauja jokių žinių apie skirstinį;
\item Tikimasi, jog MINQUE įverčiai turės mažesnį poslinkį;
\item Mokyklų duomenys pateikiami su imties svoriais.
\end{itemize}
\end{frame}
%%%%%%%%%%%%%%%%%%%%%%%%%%%%%%%%%%%%%%%%%%%%%
%%%%%%%%%%%%%%%%%%%%%%%%%%%%%%%%%%%%%%%%%%%%%
\begin{frame}
\frametitle{HLM}
\begin{itemize}
\item HLM - hierarchinis tiesinis modelis (\textit{angl. Hierarchical Linear Model});
\item Dviejų lygių modelio pavidalas: 
\[ \left\{
  \begin{array}{l}
    Y_{ij} = \beta_{0j}+\sum^P_{p = 1} \beta_{pj}\times X_{pij}+\varepsilon_{ij}; \\
    \beta_{pj} = \gamma_{p0} + \sum^{Q_p}_{q=1}\gamma_{pq}\times W_{pqj}+u_{pj};\ \forall p = 0 , \dots, P,
  \end{array} \right.\]
kur\\
\begin{scriptsize}
$j$ - j-otoji grupė;\\
$i$ - i-tasis individas j-tojoje grupėje;\\
$\beta_{0j}$ - atsitiktinis postūmis;\\
$\beta_{pj}$ - atsitiktinis posvyris;\\
$\gamma_{pq}$ - fiksuoti efektai;\\
$X_{pij}$ - pirmo lygio kintamieji;\\
$W_{pqj}$ - antro lygio kintamieji;\\
$Y_{ij}$ - aiškinamasis kintamasis.
\end{scriptsize}
\small
\item Bendras modelio pavidalas (jungtinė lygtis): $ Y_{ij} =\sum^P_{p = 1} \sum^{Q_p}_{q=1}\gamma_{pq}\times X_{pij}\times W_{pqj}+\sum^P_{p = 1} X_{pij}\times u_{pj}+\epsilon_{ij}$ arba
\begin{equation*}
Y=X\gamma+Zu+\varepsilon
\end{equation*}
\end{itemize}
\end{frame}
%%%%%%%%%%%%%%%%%%%%%%%%%%%%%%%%%%%%%%%%%%%%%
%%%%%%%%%%%%%%%%%%%%%%%%%%%%%%%%%%%%%%%%%%%%%
\begin{frame}
\frametitle{TIMSS}
\begin{itemize}
\item TIMSS - Tarptautinis matematikos ir gamtos mokslų tyrimas (\textit{angl. Trends in International Mathematics and Science Study});
\item Vykdomas kas keturis metus;
\item Dalyvauja šalys iš visų žemynų (apie 70);
\item Sudaroma dviejų pakopų sluoksninė lizdinė imtis kiekvienoje šalyje pagal jos poreikius ir TIMSS standartus;
\item Apklausiami atrinkti mokiniai ir jų tėvai, mokytojai bei mokyklos;
\item Testų rezultatai apdorojami taip, kad visų mokinių vidurkis 500 ir standatinis nuokypis 100, sudaromos 5 galimos rezultatų reikšmės (\textit{angl. plausible values});
\item Modeliuose naudojami kintamieji -- indeksai, žymimieji bei dydžio kintamieji.
\end{itemize}
\end{frame}
%%%%%%%%%%%%%%%%%%%%%%%%%%%%%%%%%%%%%%%%%%%%%
%%%%%%%%%%%%%%%%%%%%%%%%%%%%%%%%%%%%%%%%%%%%%
\begin{frame}
\frametitle{REML}
\begin{itemize}
\item REML - apribotasis didžiausio tikėtinumo metodas (\textit{angl. Restricted Maximum Likelihood}), kurį pasiūlė Harley ir Rao\footnote{Harley, H. 0., Rao, J., \textit{Maximum-likelihood estimation for the mixed analysis of variance model}, Biometrika, 54, 1967, 93}\\
\item Jei turimas modelis $Y_j=X_j\gamma+Z_ju_j+\varepsilon_j$, $u_j\sim \mathcal{N}(0_{n_j}, T)$, $\varepsilon_j\sim \mathcal{N}(0_{n_j}, \sigma^2 I_{n_j})$, kur $j=1,\dots,J$, o $Var(Y_j)=V_j=\sigma^2 I_{n_j} + Z_jTZ'_j$, tai REML įverčiai gaunami maksimizuojant \textit{log}-tikėtinumo funkciją, t.y.
\[
L(\sigma^2, T)=\sum_j L_j(\sigma^2, T) \to max,
\]
kur
\begin{small}
\[
L_j(\sigma^2, T)=-\frac{1}{2}\left( log|V_j|+(Y_j-X_j\gamma)'V_j^{-1}(Y_j-X_j\gamma)+log|X'_jV_j^{-1}X_j|\right)
\]
\end{small}
\item Fiksuotų parametrų įverčiai gaunami:
\[\hat{\gamma}=(X'\hat{V}^{-1}X)^{-1}X'\hat{V}^{-1}Y\]
\end{itemize}
\end{frame}

%%%%%%%%%%%%%%%%%%%%%%%%%%%%%%%%%%%%%%%%%%%%%
%%%%%%%%%%%%%%%%%%%%%%%%%%%%%%%%%%%%%%%%%%%%%
\begin{frame}
\frametitle{REML privalumai ir trūkumai}
Privalumai:
\begin{itemize}
\item Invariantiškumas.
\end{itemize}
Trūkumai:
\begin{itemize}
\item Negarantuoja nepaslinktumo;
\item Negarantuoja teigiamų dispersijos komponenčių įverčių;
\item Procedūra iteratyvi.
\end{itemize}
\end{frame}
%%%%%%%%%%%%%%%%%%%%%%%%%%%%%%%%%%%%%%%%%%%%%
%%%%%%%%%%%%%%%%%%%%%%%%%%%%%%%%%%%%%%%%%%%%%
\begin{frame}
\frametitle{MINQUE}
\begin{small}
\begin{itemize}
\item \indent MINQUE - mažiausios normos kvadratinis nepaslinktas įvertinys (\textit{angl. Minimum Norm Quadratic Unbiased Estimator})\footnote{Rao, C. R.  \textit{Estimation of heteroscedastic variances in linear models}, Journal of the American Statistical Association, 65, 161–172 psl., 1970};\\
 

\item \indent Jei $Var(Y)=V= \sum^l_{r=1}\theta_r Q_r$, tai $g_1 \theta_1+\dots+g_l \theta_l$ MINQUE yra $Y'AY$, jei matrica $A$ gaunama išsprendus:
\[tr(AVAV) \to min\]
su sąlygomis:
\[AX = 0\]
\[tr(AQ_r)=g_r\]
\item Tada $\hat{\theta} = (\hat{\theta_1},\dots,\hat{\theta_l})'=S^{-1}q$, kur $S=\{s_{ij}\}$, $s_{ij}=tr(CQ_iCQ_j)$, o
$q=\{q_i\}$, $q_i=Y'CQ_iCY$, $C = \hat{V}^{-1}(I-X(X' \hat{V}^{-1}X)^{-1}X \hat{V}^{-1})$, $\hat{V}=\sum^l_{r=1}\alpha_rQ_r$, $\alpha_r$ - \textit{a priori} reikšmės.

\item Fiksuotų parametrų įverčiai gaunami:
\[\hat{\gamma}=(X'\hat{V}^{-1}X)^{-1}X'\hat{V}^{-1}Y\]
\end{itemize}
\end{small}
\end{frame}
%%%%%%%%%%%%%%%%%%%%%%%%%%%%%%%%%%%%%%%%%%%%%
%%%%%%%%%%%%%%%%%%%%%%%%%%%%%%%%%%%%%%%%%%%%%
\begin{frame}
\frametitle{MINQUE privalumai ir trūkumai}
Privalumai:
\begin{itemize}
\item Nereikia žinoti skirstinio;
\item Nepaslinktumas;
\item Invariantiškumas.
\end{itemize}
Trūkumai:
\begin{itemize}
\item Priklauso nuo \textit{a priori} reikšmių;
\item Negarantuoja teigiamų dispersijos komponenčių įverčių;
\item Vertinimas reikalauja daug resursų (didelės matricos).
\end{itemize}
\end{frame}

%%%%%%%%%%%%%%%%%%%%%%%%%%%%%%%%%%%%%%%%%%%%%
%%%%%%%%%%%%%%%%%%%%%%%%%%%%%%%%%%%%%%%%%%%%%
\begin{frame}
\frametitle{Buvę tyrimai}
\begin{itemize}
\item Bagakas(1992)\footnote{ J. G. Bagakas,  \textit{ Two level nested hierarchical linear model with random intercepts
via the bootstrap}, Unpublished doctoral dissertation, Michigan State University., 1992} pritaikė MINQUE($\theta$) su saviranka dviejų lygių hierarchiniam modeliui su atsitiktiniu postūmiu; rezai???
\item Delpish(2006)\footnote{A. N. Delpish,  \textit{Comparison of Estimators in Hierarchical Linear Modeling: Restricted Maximum Likelihood Versus Bootstrap via Minimum Norm Quadratic Unbiased Estimators}, Florida State University, 2006} simuliacijų būdu parodė, jog dviejų iteracijų MINQUE(1) su saviranka duoda tikslesnius įverčius nei REML modeliui su atsitiktiniu postūmiu ir posvyriu. rezai ??
\end{itemize}
\end{frame}
%%%%%%%%%%%%%%%%%%%%%%%%%%%%%%%%%%%%%%%%%%%%%
%%%%%%%%%%%%%%%%%%%%%%%%%%%%%%%%%%%%%%%%%%%%%
\begin{frame}
\frametitle{Simuliacijų struktūra(1)}
\begin{itemize}
\item Modelis su atsitiktiniu posvyriu ir postūmiu:
\begin{equation*} \label{eq:2lvldelpish}
\left\{
\begin{array}{l}
Y_{ij} = \beta_{0j}+ \beta_{1j}\times X_{ij}+\varepsilon_{ij}; \\
\beta_{0j} = \gamma_{00} +\gamma_{01}\times W_{j}+u_{0j};\\
\beta_{1j} = \gamma_{10} +\gamma_{11}\times W_{j}+u_{1j};\\
\end{array} \right.
\end{equation*}
čia \\
\begin{small}
$\varepsilon_{ij}\sim \left(0;\sigma^2\right)$;\\
$\begin{pmatrix}
u_{0j} \\
u_{1j} \\
\end{pmatrix}\sim \left(0_2, T\right), \mathbf{T}=\begin{pmatrix}
\tau_{00} & \tau_{01} \\
\tau_{10} & \tau_{11} \\
\end{pmatrix}$; \\
$\gamma_{pq}$ - fiksuoti modelio efektai,$\ p,g = \{0,1\}$;\\
$W_j$ - antrojo lygio aiškinantysis kintamasis;\\
$X_{ij}$ - pirmojo lygio aiškinantysis kintamasis.\\ 
\ \\
\end{small}
\item 500 simuliacijų kievienam atvejui. Naudojami $\mathcal{N}$ ir $\chi^2_1$ paklaidų pasiskirstymai. Viso 36 atvejai.
\end{itemize}
\end{frame}
%%%%%%%%%%%%%%%%%%%%%%%%%%%%%%%%%%%%%%%%%%%%%
%%%%%%%%%%%%%%%%%%%%%%%%%%%%%%%%%%%%%%%%%%%%%
\begin{frame}
\frametitle{Simuliacijų struktūra(2)}

\begin{table}[!htb]
\begin{small}


\begin{subtable}{.5\linewidth}
\begin{tabular}{|c|r|}
\hline
Pavadinimas & Reikšmė\\
\hline
$\gamma_{00}$& 450  \\
$\gamma_{01}$& 10  \\
$\gamma_{10}$& 30 \\
$\gamma_{11}$& 5  \\
$\sigma^2$& 2000  \\
$X_{ij}$ &  $B\left(1; 0,2\right)$ \\
$W_{j}$ &  Indeksas \\
\hline
\end{tabular}
\end{subtable}



\begin{subtable}{.5\linewidth}
\begin{tabular}{|c|cc|}
\hline
 & \multicolumn{2}{c|}{$n_j$}\\
\hline
$m$& Nesubalansuotas & \ \ \ \ \ \ \ \ \ \ 30\ \ \ \ \ \ \ \ \ \ \\
\hline
20& P1&P2\\
35&P3&P4\\
80& P5&P6\\
\hline
\end{tabular}

\begin{tabular}{|c|cccc|}
\hline
 Pažymėjimas & $\sigma^2$&$\tau_{00}$&$\tau_{01}=\tau_{10}$&$\tau_{11}$\\
\hline
V1&2000&100&50&100\\
V2&2000&800&400&800\\
V3&2000&2000&1000&2000\\
\hline

\hline
\end{tabular}
\end{subtable}
\end{small}
\end{table}

\end{frame}
%%%%%%%%%%%%%%%%%%%%%%%%%%%%%%%%%%%%%%%%%%%%%
%%%%%%%%%%%%%%%%%%%%%%%%%%%%%%%%%%%%%%%%%%%%%
\begin{frame}
\frametitle{Simuliacijų rezultatai}

\begin{itemize}
\item Visų fiksuotų parametrų ir $\sigma^2$ įverčių santykinis poslinkis mažesnis nei 5\%;
\item Poslinkis didesnis ir mažesnis tikslumas, kai turimos $\chi^2$ paklaidos visiems metodams ir nagrinėtiems atvejams;
\item Didžiausias santykinis poslinkis dispersijos komponentėms gautas kuomet $\frac{\tau_{00}}{\sigma^2}=0,05$, nesvarbu koks paklaidų pasiskirstymas ar antro lygio subjektų skaičius;
\item MINQUE(1) anksčiau minėtu atveju poslinkis mažesnis nei REML, o įverčių išsibarstymas nedaug didesnis;
\item Beveik visais atvejais MINQUE($\theta$) įverčių poslinkis  ir išsibrstymas mažiausi, kai dizainas nesubalansuotas;
\item Kuo didesnis antro lygio objektų skaičius tuo mažesni įverčių vidutiniai santykiniai poslinkiai bei išsibarstymai ir skirtumas tarp REML ir MINQUE(1), MINQUE($\theta$).

\end{itemize}

\end{frame}
%%%%%%%%%%%%%%%%%%%%%%%%%%%%%%%%%%%%%%%%%%%%%
%%%%%%%%%%%%%%%%%%%%%%%%%%%%%%%%%%%%%%%%%%%%%
%\begin{frame}
%\frametitle{Jungtinės rezultatų lentelės fragmentas}
%
%\begin{table}
%\centering
%{\scriptsize 
%\begin{tabular}{cc|cc|cc|}
%   & & \multicolumn{2}{c|}{REML}&\multicolumn{2}{c|}{MINQUE($\theta$)}\\ \hline
% &  & CAMRBIAS & CMRSE & CAMRBIAS & CMRSE \\ 
%  \hline
%\multirow{6}{*}{P1} & \multirow{2}{*}{V1} & 0.206 & \textbf{1.155} &  \textcolor{red}{0.104} & 1.283 \\ 
%   &  & 0.254 & 1.968 & \textcolor{red}{0.115} & \textbf{1.861} \\ 
%   & \multirow{2}{*}{V2} & 0.015 & 0.23 & \textcolor{red}{0.008} & \textbf{0.212} \\ 
%   &  & 0.028 & 0.767  & \textcolor{red}{0.008} & \textbf{0.71} \\ 
%   & \multirow{2}{*}{V3} & \textcolor{red}{0.005} & 0.16  & \textcolor{red}{0.005} & \textbf{0.151} \\ 
%   &  & 0.023 & 0.625 & \textcolor{red}{0.007} & \textbf{0.588} \\ 
%   \hline \hline
%\multirow{6}{*}{P2} & \multirow{2}{*}{V1} & 0.246 & \textbf{1.116}  & \textcolor{red}{0.13} & 1.188 \\ 
%   &  & 0.249 & \textbf{2.265} & \textcolor{red}{0.155} & 2.37 \\ 
%   & \multirow{2}{*}{V2} & 0.01 & \textbf{0.212} & \textcolor{red}{0.009} & 0.214 \\ 
%   &  & \textcolor{red}{0.014} & \textbf{0.8} & 0.018 & 0.811 \\ 
%   & \multirow{2}{*}{V3} & \textcolor{red}{0.018} & \textbf{0.171} & 0.019 & 0.171 \\ 
%   &  & 0.029 & 0.778 & \textcolor{red}{0.021} & \textbf{0.708} \\ 
%\hline
%\end{tabular}
%}
%\caption{ \textcolor{red}{Raudonai} pažymėti mažesni CAMRBIAS, o \textbf{patamsinti} mažesni CRMSE. }
%\end{table}
%
%\centering
%{
%$MRBIAS=\frac{1}{S}\sum_{i=1}^S\frac{\hat{\theta}_i-\theta}{\theta}$;
%$MRSE=\frac{1}{S}\sum_{i=1}^S\left(\frac{\hat{\theta}_{i}-\theta}{\theta}\right)^2$;\\
%$CAMRBIAS=\frac{1}{n_{\theta}}\sum_1^{n_{\theta}}|MRBIAS_{\theta}|;CMRSE=\frac{1}{n_{\theta}}\sum_1^{n_{\theta}}MRSE_{\theta }$
%
%}
%
%
%\end{frame}
%%%%%%%%%%%%%%%%%%%%%%%%%%%%%%%%%%%%%%%%%%%%%
%%%%%%%%%%%%%%%%%%%%%%%%%%%%%%%%%%%%%%%%%%%%%
\begin{frame}
\frametitle{PWMINQUE ir simuliacijos}
\begin{itemize}
\item PWMINQUE -- tikimybėmis pasvertas MINQUE metodas;
\item Svėrimo metodas pritaikytas pagal PWIGLS metodą\footnote{D. Pfeffermann, C. J. Skinner, D. J. Holmes, H. Goldstein, J. Rasbash, Weighting for unequal selection probabilities in multilevel models, \textit{Journal of Royal Statistical Society}, Series B, 1998, 60(l), p. 23 - 40.};
\item Simuliacijoms naudota Steele ir kt. stuktūra\footnote{F. Steele, P. Clarke, H. Goldstein, \textit{Weighting in MLwiN}, Centre for Multilevel Modeling (CML), Bristol: University of Bristol, 2011};
%\begin{equation*}\label{eq:wsimul}
%\left\{
%\begin{array}{l}
%\begin{split}
%Y_{ij}&=\beta_{0j}+\beta_{1j}X_{1ij}+\varepsilon_{ij}, &\varepsilon_{ij}\sim \mathcal{N}(0, \sigma^2);\\
%\beta_{0j}&=\gamma_{00}+\gamma_{01}X_{2j}+u_{0j}, & u_{0j}\sim \mathcal{N}(0, \tau_{00});\\
%\beta_{1j}&=\gamma_{10}+u_{1j}, & u_{1j}\sim \mathcal{N}(0, \tau_{11}).
%\end{split}
%\end{array} \right.
%\end{equation*}
%\begin{small}
%čia $\sigma^2=1$ ir 
% $\mathbf{T}=\begin{pmatrix}
%\tau_{00} & \tau_{01} \\
%\tau_{10} & \tau_{11} \\
%\end{pmatrix}=\begin{pmatrix}
%\frac{1}{4}& \frac{\sqrt{3}}{8} \\
% \frac{\sqrt{3}}{8} & \frac{3}{4}\\
%\end{pmatrix}$;\\
% $\gamma_{00}=\gamma_{01}=\gamma_{10}=1$;\\
%$X_2\sim \mathcal{B}(1;0,5)$;\\
%$X_1\sim \mathcal{N}(0,1)$ - centruotas pagal $j$.
%\end{small}
\item Modelis vertintas REML, MINQUE(1), PWMINWUE(1), MINQUE($\theta$) ir PWMINQUE($\theta$) metodais;
\item PWMINWUE(1) ir PWMINQUE($\theta$) metodais gautas vidutinis santykinis poslinkis akivaizdžiai mažesnis, išsibarstymas didesnis.
\end{itemize}
\end{frame}
%%%%%%%%%%%%%%%%%%%%%%%%%%%%%%%%%%%%%%%%%%%%%
%%%%%%%%%%%%%%%%%%%%%%%%%%%%%%%%%%%%%%%%%%%%%
%\begin{frame}
%\frametitle{PWMINQUE ir simuliacijos(2)}
%\begin{itemize}
%\item Sugeneruojama $M=300,\ N_j=30$. Parenkama $ m=35$, $n_j=20$ su tikimybėmis:
%\begin{equation*}
%\pi_j=
%\left\{
%\begin{array}{l}
%0,225, \text{ kai } u_{0j} \notin \left(\bar{u}_0-2\times sd_0; \bar{u}_0+2\times sd_0\right) \text{ ir } |u_{1j}| > 1;\\
%0,425, \text{ kai } u_{0j} \in \left(\bar{u}_0-2\times sd_0; \bar{u}_0+2\times sd_0\right) \text{ ir } |u_{1j}| > 1;\\
%0,525, \text{ kai } u_{0j} \notin \left(\bar{u}_0-2\times sd_0; \bar{u}_0+2\times sd_0\right) \text{ ir } |u_{1j}| \leq 1;\\
%0,725, \text{ kai } u_{0j} \in \left(\bar{u}_0-2\times sd_0; \bar{u}_0+2\times sd_0\right) \text{ ir } |u_{1j}| \leq 1,
%\end{array} \right.
%\end{equation*}
%ir
%\begin{equation*}
%\pi_{i|j}=
%\left\{
%\begin{array}{l}
%0,25, \text{ kai } \varepsilon_{ij}>0;\\
%0,75, \text{ kai } \varepsilon_{ij}\leq0.\\
%
%\end{array} \right.
%\end{equation*}
%\item Gaunami imties svoriai $w_j=\frac{m}{\pi_j}$, $w_{i|j}=\frac{n_j}{\pi_{i|j}}$ ir $w_{ij}=w_j\times w_{i|j}$.
%\end{itemize}
%\end{frame}
%%%%%%%%%%%%%%%%%%%%%%%%%%%%%%%%%%%%%%%%%%%%%
%%%%%%%%%%%%%%%%%%%%%%%%%%%%%%%%%%%%%%%%%%%%%
\begin{frame}[shrink=20]
\frametitle{TIMSS galutis modelis}
\begin{equation*} \label{eq:timss2}
\left\{
\begin{array}{l}
\begin{split}
MatRes_{ij}&=\beta_{0j}+\beta_{1j}\times SSEX_{ij}+\beta_{2j} \times STHMWRK_{ij}+\beta_{3j}\times CSMCONF_{ij}+\\
&+\beta_{4j}\times CSHEDRES_{ij}+\varepsilon_{ij};\\
\beta_{0j}&=\gamma_{00}+\gamma_{01}\times MVIETA.2_j+\gamma_{02}\times MSUCCESS_j+\\
&+\gamma_{03}\times MDYDIS_j+u_{0j};\\
\beta_{1j}& = \gamma_{10};\\
\beta_{2j}&=\gamma_{20};\\
\beta_{3j}&=\gamma_{30}+u_{3j};\\
\beta_{4j}&=\gamma_{40},\\
\end{split}
\end{array} \right.
\end{equation*}

\noindent čia $SSEX$ -- žymimasis, mokinio lytis (0 -- mergaitė, 1 -- berniukas);\\
$STHMWRK$ -- žymimasis, mokinio skiriamas laikas namų darbams (0 -- 45 min ir mažiau, 1 -- daugiau nei 45 min);\\
$CSMCONF$ -- indeksas,  mokinio pasitikėjimas savimi atliekant matematikos užduotis, centruotas pagal j;\\
$CSHEDRES$ -- indeksas, mokinio namų mokymosi ištekliai, centruotas pagal j;\\
$MVIETA.2$ -- žymimasis, mokyklos vietovė (1 -- mažas miestelis);\\
$MSUCCESS$ -- indeksas, mokyklos skiriamas dėmesys akademinei sėkmei;\\
$MDYDIS$ -- skaičius, mokyklos aštuntokų skaičius.\\
\end{frame}
%%%%%%%%%%%%%%%%%%%%%%%%%%%%%%%%%%%%%%%%%%%%%
%%%%%%%%%%%%%%%%%%%%%%%%%%%%%%%%%%%%%%%%%%%%%
\begin{frame}
\frametitle{TIMSS galutio modelio įverčiai}
\resizebox{\linewidth}{!}{
\begin{tabular}{|l|c|c|c|c|c|}
\hline
Parametras & REML & MINQUE(1) & PWMINQUE(1) & MINQUE($\theta$) & PWMINQUE($\theta$)\\
\hline
$\gamma_{00}$ & 471.913 & 481.680 & 399.091 & 471.271 & 444.265 \\ 
$\gamma_{01}$ {\footnotesize  (MVIETA.2)}& -27.960 & -29.294 & -80.377 & -27.889 & -26.661 \\ 
$\gamma_{02}$ {\footnotesize (MSUCCESS)} & 1.916 & 1.134 & 16.545 & 1.965 & 5.120 \\ 
$\gamma_{03}$  {\footnotesize(MDYDIS)}& 0.018 & 0.015 & -0.063 & 0.018 & 0.006 \\ 
$\gamma_{10}$ {\footnotesize (SSEX)} & 11.112 & 11.090 & 11.903 & 11.112 & 13.061 \\ 
$\gamma_{20}$  {\footnotesize(STHMWRK)}& 6.539 & 6.521 & -8.669 & 6.533 & 4.518 \\ 
$\gamma_{30}$  {\footnotesize(CSMCONF)}& 18.579 & 18.551 & 21.220 & 18.580 & 19.102 \\ 
$\gamma_{40}$  {\footnotesize(CSHEDRES)}& 9.518 & 9.328 & 11.168 & 9.525 & 10.951 \\ 
$\sigma^2$ &2843.308 & 2842.028 & 2858.521 & 2843.560 & 2890.473 \\
$\tau_{00}$ & 946.783 & 998.554 & 853.321 & 928.238 & 704.666 \\ 
$\tau_{01}$ & -92.772 & -82.178 & -141.233 & -91.622 & -106.479 \\ 
$\tau_{11}$ & 21.711 & 1.053 & -12.328 & 22.754 & 25.003 \\
\hline
$R_1^2$ & 0.516 & 0.504 & -12.251 & 0.517 & 0.515\\
\hline
\end{tabular}}
\end{frame}
%%%%%%%%%%%%%%%%%%%%%%%%%%%%%%%%%%%%%%%%%%%%%
%%%%%%%%%%%%%%%%%%%%%%%%%%%%%%%%%%%%%%%%%%%%%
%\begin{frame}
%\frametitle{Atlikta šiame darbe}
%\begin{itemize}
%\item Matricinis dviejų lygių HLM pavidalas MINQUE;
%\item $R$ funkcijos dviejų lygių HLM modeliams vertinti MINQUE;
%\item Simuliacijų būdu palyginti REML ir MINQUE(0), MINQUE(1) bei MINQUE($\theta$) per RBIAS (santykinis poslinkis) ir RMSE (santykinė kvadratinė paklaida);
%\item Alternatyvus metodas WMINQUE vertinti su svoriais (PWIGLS);
%\item Simuliacijų būdu parodyta, jog WMINQUE tikslesnis vertinant modelius su imties svoriais.
%\end{itemize}
%\end{frame}
%%%%%%%%%%%%%%%%%%%%%%%%%%%%%%%%%%%%%%%%%%%%%
%%%%%%%%%%%%%%%%%%%%%%%%%%%%%%%%%%%%%%%%%%%%%
\begin{frame}
\frametitle{Apibendrinimas}
\begin{itemize}
\item Sukurtas lankstus R funkcijų rinkinys dviejų lygių HLM vertinimui MINQUE metodu;
\item Pritaikytas PWIGLS svėrimo metodas MINQUE metodui;
\item Simuliacijų būdu patikrinta, jog REML metodo įverčiai dviejų lygių HLM nėra reikšmingai labiau paslinkti nei MINQUE kuomet turimas pakankamai didelis (m$\geq 35$) antrojo lygio objektų skaičius ir $\frac{\tau_{00}}{\sigma^2}\geq 0.4$;
\item Beveik visais atvejais REML metodu gauti įverčiai stabilesni, kai turimas subalansuotas dizainas.

\end{itemize}
\end{frame}

%%%%%%%%%%%%%%%%%%%%%%%%%%%%%%%%%%%%%%%%%%%%%
%%%%%%%%%%%%%%%%%%%%%%%%%%%%%%%%%%%%%%%%%%%%%
\begin{frame}
\frametitle{Pabaiga}
\huge
Ačiū už dėmesį
\end{frame}
\end{document}

%\begin{table}
%\centering
%{\footnotesize
%\begin{tabular}{cc|cc|cc|}
%   & & \multicolumn{2}{c|}{REML}&\multicolumn{2}{c|}{MINQUE($\theta$)}\\ \hline
% &  & CAMRBIAS & CRMSE & CAMRBIAS & CRMSE \\ 
%  \hline
%\multirow{6}{*}{P1} & \multirow{2}{*}{V1} & 0.206 & \textbf{1.155} &  \textcolor{red}{0.104} & 1.283 \\ 
%   &  & 0.254 & 1.968 & \textcolor{red}{0.115} & \textbf{1.861} \\ 
%   & \multirow{2}{*}{V2} & 0.015 & 0.23 & \textcolor{red}{0.008} & \textbf{0.212} \\ 
%   &  & 0.028 & 0.767  & \textcolor{red}{0.008} & \textbf{0.71} \\ 
%   & \multirow{2}{*}{V3} & \textcolor{red}{0.005} & 0.16  & \textcolor{red}{0.005} & \textbf{0.151} \\ 
%   &  & 0.023 & 0.625 & \textcolor{red}{0.007} & \textbf{0.588} \\ 
%   \hline \hline
%\multirow{6}{*}{P2} & \multirow{2}{*}{V1} & 0.246 & \textbf{1.116}  & \textcolor{red}{0.13} & 1.188 \\ 
%   &  & 0.249 & \textbf{2.265} & \textcolor{red}{0.155} & 2.37 \\ 
%   & \multirow{2}{*}{V2} & 0.01 & \textbf{0.212} & \textcolor{red}{0.009} & 0.214 \\ 
%   &  & \textcolor{red}{0.014} & \textbf{0.8} & 0.018 & 0.811 \\ 
%   & \multirow{2}{*}{V3} & \textcolor{red}{0.018} & \textbf{0.171} & 0.019 & 0.171 \\ 
%   &  & 0.029 & 0.778 & \textcolor{red}{0.021} & \textbf{0.708} \\ 
%\hline
%\end{tabular}
%}
%\caption{ \textcolor{red}{Raudonai} pažymėti mažesni CAMRBIAS, o \textbf{patamsinti} mažesni CRMSE. }
%\end{table}
%    \end{column}
%    \begin{column}{0.4\textwidth}
%$MRBIAS_{\theta MVP}=\frac{1}{S}\sum_1^S\frac{\hat{\theta}_{MVP}-\theta}{\theta}$; $MRMSE_{\theta MVP}=\frac{1}{S}\sum_1^S\left(\frac{\hat{\theta}_{MVP}-\theta}{\theta}\right)^2$; $CAMRBIAS_{MVP}=\frac{1}{n_{\theta}}\sum_1^{n_{\theta}}|MRBIAS_{\theta MVP}|$;$CMRMSE_{MVP}=\frac{1}{n_{\theta}}\sum_1^{n_{\theta}}MRMSE_{\theta MVP}$