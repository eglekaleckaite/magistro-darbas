\documentclass[utf8,hyperref={unicode,pdftex}]{beamer}
%\documentclass[utf8]{beamer}

\mode<presentation>
{
\usetheme{Warsaw}
}
\setbeamertemplate{navigation symbols}{}

\usepackage[utf8]{inputenc}
\usepackage[L7x]{fontenc}
\usepackage[lithuanian]{babel}
\usepackage{lmodern}
\usepackage{amsmath}
\usepackage{amssymb}
\usepackage{bm}
\usepackage{graphicx}
\usepackage{multirow}

\beamersetuncovermixins{\opaqueness<1>{25}}{\opaqueness<2->{15}}

\title[\hspace{130pt} p. \insertpagenumber\enspace iš \insertdocumentendpage\enspace ]{HLM modeliai TIMSS duomenims}
\author[ E. Kaleckaitė]{Eglė Kaleckaitė\\
Vadovas: Vydas Čekanavičius}
\institute{Vilniaus Universitetas, Matematikos ir Informatikos Fakultetas}
\date{2013 gruodžio 11d.}
\begin{document}
\begin{frame}
\titlepage
\end{frame}
\begin{frame}
\frametitle{Turinys}
\Large
\begin{enumerate}
\item Įvadas: TIMSS ir HLM
\item TIMSS Lietuvos atveju
\item Turimi duomenys ir jų problemos
\item Vertinimo procedūros
\item Rezultatai ir išvados
\end{enumerate}
\end{frame}
%\section{Įvadas}
\begin{frame}
\frametitle{Įvadas: TIMSS}
\begin{itemize}
\item TIMSS - Tarptautinis matematikos ir gamtos mokslų tyrimas (\textit{angl. Trends in International Mathematics and Science Study});
\item IEA - Tarptautinė švietimo pasiekimų vertinimo asociacija (\textit{angl. International Association of the Evaluation of Educational Achievement});
\item Vykdomas kas keturis metus;
\item Dalyvauja šalys iš visų žemynų (apie 70);
\item Sudaroma dviejų pakopų sluoksninė lizdinė imtis kiekvienoje šalyje pagal jos poreikius ir TIMSS standartus;
\item Apklausiami atrinkti mokiniai ir jų tėvai, mokytojai bei mokyklos.
\end{itemize}
\end{frame}

\begin{frame}
\frametitle{Įvadas: HLM}
\begin{itemize}
\item HLM - hierarchinis tiesinis modelis (\textit{angl. Hierarchical Linear Model});
\item Dviejų lygių modelio pavidalas: 
\[ \left\{
  \begin{array}{l}
    Y_{ij} = \beta_{0j}+\sum^P_{p = 1} \beta_{pj}\times X_{pij}+\varepsilon_{ij}; \\
    \beta_{pj} = \gamma_{p0} + \sum^{Q_p}_{q=1}\gamma_{pq}\times W_{pqj}+u_{pj};\ \forall p = 0 , \dots, P,
  \end{array} \right.\]
kur\\
\begin{small}
$j$ - j-otoji grupė;\\
$i$ - i-tasis individas j-tojoje grupėje;\\
$\beta_{0j}$ - atsitiktinis postūmis;\\
$\beta_{pj}$ - atsitiktinis posvyris;\\
$\gamma_{pq}$ - fiksuoti efektai;\\
$X_{pij}$ - pirmo lygio kintamieji;\\
$W_{pqj}$ - antro lygio kintamieji;\\
$Y_{ij}$ - aiškinamasis kintamasis.
\end{small}
\small
\item Bendras modelio pavidalas (jungtinė lygtis):
\[ 
    Y_{ij} =\sum^P_{p = 1} \sum^{Q_p}_{q=1}\gamma_{pq}\times X_{pij}\times W_{pqj}+\sum^P_{p = 1} X_{pij}\times u_{pj}+\epsilon_{ij}; 
 \]
\end{itemize}
\end{frame}

\begin{frame}
\frametitle{TIMSS Lietuvoje}
\begin{itemize}
\item Lietuvoje TIMSS tyrimą vykdo ir pagrindines analizes bei suvestines atlieka \textit{Nacionalinis egzaminų centras};
\item J. Dudaitė. \textit{Pagrindinės mokyklos matematikos mokymosi rezultatų pokyčių vertinimas edukacinės paradigmos virsmo sąlygomis}, Daktaro disertaciją, 2008
\item G. Akyuz, G. Berberoglu, \textit{Teacher and Classroom Characteristics and Their Relations to Mathematics Achievement of the Students in the TIMSS}, New Horizons in Education, 2010
\end{itemize}
\end{frame}

\begin{frame}
\frametitle{J. Dudaitės sudaryti modeliai}
HLM modeliai buvo sudaryti atskirai kiekvienam kintamajam:
\[ \left\{
  \begin{array}{l l}
    Y_{ij} = \beta_{0j}+\varepsilon_{ij}; \\
    \beta_{0j} = \gamma_{00} + \gamma_{01}\times W_{j}+u_{0j};
  \end{array} \right.\]
\small
čia $i$ - žymi $i$-tajį mokinį,\\
$j$ - $j$-tają mokyklą/klasę/mokytoją,\\
$Y_{ij}$ - $i$-tojo mokinio iš $j$-tosios mokyklos/klasės/mokytojo matematinio raštingumo rezultatas,\\
$\beta_{0j}$ - $j$-tosios mokyklos/klasės/mokytojo įnašas,\\
$r_{ij}$ - pirmo lygio paklaidos,\\
$\gamma_{00}$ - vidutinis mokyklų/klasių/ mokytojų įnašas,\\
$\gamma_{01}$ - vidutinis mokyklų/klasių/ mokytojų nuolydis,\\
$W_j$ - antro lygio kintamasis, \\
$u_{0j}$ - antro lygio paklaidos.
\end{frame}

\begin{frame}
\frametitle{Akyuz ir Berberoglu sudarytas modelis}
Šie autoriai sudarė tik vieną modelį:
\[ \left\{
  \begin{array}{l}
    Y_{ij} = \beta_{0j}+\beta_{1j}\times X_{ij}+\varepsilon_{ij}; \\
    \beta_{0j} = \gamma_{00} + \gamma_{01}\times W_{1j}+\dots+\gamma_{0m}\times W_{mj}+u_{0j};\\
    \beta_{1j} = \gamma_{10} + u_{1j};
  \end{array} \right.\]
\small
čia $i$ - žymi $i$-tajį mokinį,\\
$j$ - $j$-tają mokyklą/klasę/mokytoją,\\
$Y_{ij}$ - $i$-tojo mokinio iš $j$-tosios mokyklos/klasės/mokytojo matematinio raštingumo rezultatas,\\
$\beta_{0j}$ - $j$-tosios mokyklos/klasės/mokytojo įnašas,\\
$\beta_{1j}$ - atsitiktinis posvyris,\\
$X_{ij}$ - pirmo lygio kintamasis, šiuo atveju, namų edukaciniai ištekliai, \\
$r_{ij}$ - pirmo lygio paklaidos,\\
$\gamma_{00}$ - vidutinis mokyklų/klasių/ mokytojų įnašas,\\
$\gamma_{01}$ - vidutinis mokyklų/klasių/ mokytojų nuolydis,\\
$W_{mj}$ - antro lygio kintamasis, \\
$u_{0j}$ - atsitiktinis efektas,\\
$u_{1j}$ - atsitiktinis efektas.
\end{frame}

\begin{frame}
\frametitle{Bendresnis HLM pavidalas bei modelio prielaidos (1)}
Tegu turime $N$ pirmo lygio individų, kurie yra natūraliai suskaidyti į $J$ grupių (mokyklų). Kiekvienoje iš $J$ grupių yra $n_j$ individų, kur $\sum^J_{j=1} n_j = N$.
\[\mathbb{Y}_j=\mathbb{X}_j\beta_j+\varepsilon_j;
\]
Antrame lygyje:
\[ \beta_{jk}=\omega'_{jk}\gamma_{j}+u_{jk},
\]
čia
\[
\mathbb{W}_{j} =
 \begin{pmatrix}
 \omega'_{j1}  &0 & \cdots &0 \\
 0 &\omega'_{j2} & \cdots & 0 \\
  \vdots  & \vdots  & \ddots & \vdots  \\
  0 &0 & \cdots & \omega'_{jq}
 \end{pmatrix} = \omega'_{j1}\oplus \omega'_{j2}\oplus \cdots \oplus \omega'_{jq};
\]
\[ \gamma = (\gamma'_1, \gamma'_2, \dots, \gamma'_q)'; \ u_j = (u_{j1}, u_{j2}, \dots, u_{jq})'\]
Tada
\[\beta_j=\mathbb{W}_j\gamma+u_j\]
\end{frame}

\begin{frame}
\frametitle{Bendresnis HLM pavidalas bei modelio prielaidos (2)}
Įstatome $\beta_j$ 
\begin{large}
\begin{equation}
\mathbb{Y}_j=\mathbb{X}_j \mathbb{W}_j\gamma+\mathbb{X}_j u_j+\varepsilon_j
\end{equation}
\end{large}
Prielaidos:
\begin{itemize}
\item $\varepsilon_{ij} \sim \mathcal{N}(0, \sigma^2\mathbb{I}_{n_j})$
\item $Cov(X_{qj},\varepsilon_{qj})=0, \forall q$ 
\item $u_j = (u_{j1}, u_{j2}, \dots, u_{jq})' \overset{iid}{\sim} \mathcal{N}(0,T)$
\item $Cov(\mathbb{W}_{j}, u_j)=0$
\item $Cov(\varepsilon_j, u_j)=0$
\item $Cov(\mathbb{X}_j, u_j)=0$
\item $Cov(\mathbb{W}_j, \varepsilon_j)=0$
\end{itemize}
\end{frame}

\begin{frame}
\frametitle{Bendresnis HLM pavidalas bei modelio prielaidos (3)}
\small
Tokį modelį galima perrašyti standartiniu dispersijos komponenčių modelio pavidalu:
\[
\mathbf{Y}_j=\begin{pmatrix}
Y_{1j}\\
Y_{2j}\\
\cdots\\
Y_{n_jj}
 \end{pmatrix} ;
\mathbf{X}_j=
\begin{pmatrix}
X_{1j}W_j\\
X_{2j}W_j\\
\cdots\\
X_{n_jj}W_j
 \end{pmatrix} ;
\mathbf{Z}_{j} =
 \begin{pmatrix}
X_{j1}   \\
X_{j2}  \\
  \cdots  \\
X_{n_jj}
 \end{pmatrix};
\boldsymbol{\varepsilon}_{j} =
 \begin{pmatrix}
\varepsilon_{j1}   \\
\varepsilon_{j2}  \\
  \cdots  \\
\varepsilon_{n_jj}
 \end{pmatrix} 
\]
\[
\mathbf{Y}=\begin{pmatrix}
Y_{1}\\
Y_{2}\\
\cdots\\
Y_{J}
 \end{pmatrix} ;
\mathbf{X}=
\begin{pmatrix}
\mathbf{X}_{1}\\
\mathbf{X}_{2}\\
\cdots\\
\mathbf{X}_{J}
 \end{pmatrix} ;
\mathbf{Z} =
 \begin{pmatrix}
\mathbf{Z}_{1} &0&\cdots&0  \\
0& \mathbf{Z}_{2}&\cdots&0  \\
  \vdots  & \vdots  & \ddots & \vdots  \\
0&0&\cdots& \mathbf{Z}_J
 \end{pmatrix};
\mathbf{u}=
 \begin{pmatrix}
u_{1}   \\
u_{2}  \\
  \cdots  \\
u_{J}
 \end{pmatrix} 
\boldsymbol{\varepsilon}=
 \begin{pmatrix}
\varepsilon_{1}   \\
\varepsilon_{2}  \\
  \cdots  \\
\varepsilon_J
 \end{pmatrix} 
\]
\large
\begin{equation}
\mathbf{Y}=\mathbf{X}\boldsymbol{\gamma}+\mathbf{Z}\mathbf{u}+\boldsymbol{\varepsilon}
\end{equation}
\small
\[E(\mathbf{Y}) = \mathbf{X}\boldsymbol{\gamma}; Var(\mathbf{Y})=\mathbf{V}=\mathbf{Z}T\mathbf{Z}'+\sigma^2\mathbf{I}\]
\end{frame}

%
%\begin{frame}
%\frametitle{Vertinimo procedūra: ML, REML}
%
%\end{frame}


\begin{frame}
\frametitle{Problemos}
\large
\begin{itemize}
\item Mokyklų duomenims normalumas dažnai negali būti garantuotas\footnote{Raudenbush, S. W., and Bryk, A. S.  \textit{Hierarchical Linear Models: Applications and
Data Analysis Methods}, (2nd ed.), 2002};
\item Dėl TIMSS struktūros reikalingas vertinimas su imties svoriais;
\item TIMSS pateikia 5 galimas testo atlikimo reikšmes kiekvienam mokiniui.
\end{itemize}
\end{frame}

\begin{frame}
\frametitle{Problemos: Nenormalumas}
\includegraphics[height=7cm]{teorkv.pdf}
\end{frame}

\begin{frame}
\frametitle{MINQUE}
\begin{itemize}
\item MINQUE - mažiausios normos kvadratinis nepaslinktas įvertinys (\textit{angl. Minimum Norm Quadratic Unbiased Estimator}), kurį pasiūlė Rao(1970)\footnote{Rao, C. R.  \textit{Estimation of heteroscedastic variances in linear models}, Journal of the American Statistical Association, 65, 161–172 psl., 1970};
\item Bagakas(1992)\footnote{Bagakas, J. G  \textit{ Two level nested hierarchical linear model with random intercepts
via the bootstrap}, Unpublished doctoral dissertation, Michigan State University., 1992} pritaikė IMINQUE su saviranka dviejų lygių hierarchiniam modeliui su atsitiktiniu postūmiu;
\item Delpish(2006)\footnote{Delpish, A. N.  \textit{Comparison of Estimators in Hierarchical Linear Modeling: Restricted Maximum Likelihood Versus Bootstrap via Minimum Norm Quadratic Unbiased Estimators}, Florida State University, 2006} simuliacijų būdu parodė, jog IMINQUE su saviranka duoda mažesnes standartines paklaidas nei REML.
\end{itemize}
\end{frame}

\begin{frame}
\frametitle{MINQUE privalumai ir trūkumai}
Privalumai:
\begin{itemize}
\item Nereikalauja duomenų normalumo;
\item Nereikia žinoti skirstinio;
\item Nepaslinktumas;
\item Invariantiškumas.
\end{itemize}
Trūkumai:
\begin{itemize}
\item Priklauso nuo \textit{a priori} reikšmių;
\item Negarantuoja teigiamų dispersijos įverčių;
\item Vertinimas reikalauja daug resursų (didelės matricos).
\end{itemize}
\end{frame}



\begin{frame}
\frametitle{MINQUE procedūra}
Pasak Rao, jei $Y$ dispersiją galima išreikšti tiesine nežinomų parametrų ir žinomų matricų kombinacija $V= \sum^l_{r=1}\theta_r Q_r$, tai tiesinės kombinacijos $g_1 \theta_1+\dots+g_l \theta_l$ MINQUE yra $Y'AY$, jei matrica $A$ gaunama išsprendus:
\[tr(AVAV) \to min\]
su sąlygomis:
\[AX = 0\]
\[tr(AQ_r)=g_r\]
Tada $\hat{\theta} = (\hat{\theta_1},\dots,\hat{\theta_l})'=S^{-1}q$, kur $S=\{s_{ij}\}$, $s_{ij}=tr(CQ_iCQ_j)$, o
$q=\{q_i\}$, $q_i=Y'CQ_iCY$, $C = \hat{V}^{-1}(I-X(X' \hat{V}^{-1}X)^{-1}X \hat{V}^{-1})$, $\hat{V}=\sum^l_{r=1}\alpha_rQ_r$, $\alpha_r$ - \textit{a priori} reikšmės.

Fiksuotų parametrų įverčiai gaunami pagal GLS:
\[\hat{\gamma}=(X'\hat{V}^{-1}X)^{-1}X'\hat{V}^{-1}Y\]
\end{frame}

\begin{frame}
\frametitle{MINQUE procedūra dviejų lygių HLM}
%Turėdami HLM išraišką iš anksčiau $Y=X\gamma+Zu+\varepsilon$, galim $Z$ išreikšti kaip $Z=(\tilde{Z_1}\vdots\cdots\vdots\tilde{Z_J})$ ir atitinkamai $u'=(u_1'\cdots u_J')$.
%Tada $V=\sum^J_{j=1}\tilde{Z_j}T\tilde{Z_j}'+\sigma^2I$.\\
Turnti $V=ZTZ'+\sigma^2I$ , $T=\{\tau_{ij}\}$ galima išskaidyti  į $\sum^l_{r=1}\theta_rT_r$, kur $\theta = (\tau_{00}, \tau_{01}, \dots, \tau_{0q}, \tau_{11},\dots, \tau_{q-1,q})'$ ir
\small
\[
T_1=
 \begin{pmatrix}
1&0&\cdots&0   \\
0&0&\cdots&0  \\
\vdots&\vdots&  \cdots &\vdots \\
0&0&\cdots&0
 \end{pmatrix};
T_2=
 \begin{pmatrix}
0&1&\cdots&0   \\
1&0&\cdots&0  \\
\vdots&\vdots&  \cdots &\vdots \\
0&0&\cdots&0
 \end{pmatrix};
\dots;
T_l=
 \begin{pmatrix}
0&0&\cdots&0   \\
0&0&\cdots&0  \\
\vdots&\vdots&  \cdots &\vdots \\
0&0&\cdots&1
 \end{pmatrix}
\]
Tada $Q_r= ZT_rZ'$.\\
Tegul $\theta_0 = \sigma^2$ ir $Q_0=I$, tuomet 
\large
\[V=\sum^l_{r=0} \theta_rQ_r\]

\end{frame}

\begin{frame}
\frametitle{Problemos: Svoriai}
Mokyklų svoriai: 
\begin{equation*}
W^{sc}_{hj} = \frac{M_h}{m_{hj}n_h}; M_h=\sum^{N_h}_{j=1} m_{hj}
\end{equation*}
Klasių svoriai iš $h$-tojo lizdo $j$-tosios mokyklos: 
\begin{equation*}
W^{cl}_{k|hj} = \frac{C_{hj}}{c_{hj}}
\end{equation*}
Mokinių svoriai:
\begin{equation*}
W^{st}_{i|hjk} = 1
\end{equation*}
Bendras $i$-tojo mokinio svoris:
\begin{equation}
W^{st}_{ikjh} = W^{sc}_{hj}\times W^{cl}_{k|hj}
\end{equation}
\small
čia $M_h$ - bendras aštuntokų skaičius $h$-tajame lizde, $m_{hj}$ - aštuntokų skaičius $j$-tojoje $h$-tojo lizdo mokykloje, $n_h$ - mokyklų, parinktų iš $h$-tojo lizdo skaičius, $N_h$ - mokyklų skaičius $h$-tajame lizde, $C_{hj}$ - klasių skaičius $j$-tojoje mokykloje,
$c_{hj}$ - parintų klasių skaičius $j$-tojoje mokykloje.

\end{frame}

\begin{frame}
\frametitle{PWIGLS}
\begin{itemize}
\item PWIGLS - tikimybėmis pasvertas apibendrintas mažiausių kvadratų metodas (\textit{angl. Probability Weighted Generalized Least Squares}), pasiūlytas Pfeffermann ir kitų (1998)\footnote{Pfeffermann, D., Skinner, C. J., Holmes, J., Goldstein, H., Rasbash, J., \textit{Weighting for Unequal Selection Probabilities in Multilevel Models}, Journal of the Royal Statistical Society, 23 - 40 psl.,1998}
\item Privalumai:
\begin{itemize}
\item Atsižvelgiama į imties dizainą;
\item Geresnės asimptotinės savybės nei IGLS;
\end{itemize}
\item Trūkumai:
\begin{itemize}
\item Sudarytas tik dviejų lygių modeliui;
\item Reikalauja normalumo;
\item Tinka tik tolydiems aiškinantiesiems kintamiesiems.
\end{itemize}
\end{itemize}
\end{frame}

\begin{frame}
\frametitle{IGLS procedūra dviejų lygių HLM}
Ši procedūra, praktiškai sutampa su IMINQUE. Tik reikia pasdtebėti, jog $V = V_1\oplus V_2\oplus\dots\oplus V_J$, kur $V_j=Z_jTZ_j'+\sigma^2I_{n_j}$. Tuomet $Q_r=Q_{1r}\oplus Q_{2r} \oplus \dots \oplus Q_{Jr}$

Todėl 
\[\hat{\theta}=S^{-1}q,\] kur
\[s_{kl}=\sum^J_{j=1}tr\left(\hat{V}^{-1}_j Q_{jk}\hat{V}^{-1}_j Q_{jl}\right),\]
\[q_k=\sum^J_{j=1}tr\left(\hat{V}^{-1}_j Q_{jk}\hat{V}^{-1}_j E_{jj}(\hat{\gamma})\right),\]
\[E_{jj}(\hat{\gamma})=(Y_j-X_j\hat{\gamma})(Y_j-X_j\hat{\gamma})',
\]
\[\hat{\gamma}=(X'\hat{V}^{-1}X)^{-1}X'\hat{V}^{-1}Y.\]
\end{frame}

\begin{frame}
\frametitle{PWIGLS procedūra dviejų lygių HLM}
Tegul turimi pirmo lygio svoriai $w_{i|j}$, antro lygio svoriai $w_j$ ir $w_{ij}=w_{i|j}\times w_j$.
Pagal Pfeffermann\footnote{Pfeffermann, D., Skinner, C. J., Holmes, J., Goldstein, H., Rasbash, J., \textit{Weighting for Unequal Selection Probabilities in Multilevel Models}, Journal of the Royal Statistical Society, 23 - 40 psl.,1998}, užtenka kiekvieną $Z_j$ padauginti iš $w^{-\frac{1}{2}}_j$ ir vietoje $I_{n_j}$ matricos nauditi $D_j=diag(w^{-1}_{1j}, w^{-1}_{2j}, \dots , w^{-1}_{n_jj})$. Tada belieka pakeisti 

\[s_{kl}=\sum^J_{j=1}w_j tr\left(\hat{V}^{-1}_j Q_{jk}\hat{V}^{-1}_j Q_{jl}\right).\]

\end{frame}


\begin{frame}
\frametitle{Pabaiga}
\huge
Ačiū už dėmesį
\end{frame}
\end{document}

