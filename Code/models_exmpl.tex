\documentclass[a4paper]{article}
\usepackage{ucs}
\usepackage[utf8x]{inputenc}
\usepackage[L7x]{fontenc}
\usepackage[T1]{fontenc}
\usepackage[lithuanian]{babel}
\title{HLM modeliai TIMSS duomenims\\Pavyzdžiai}
\author{Eglė Kaleckaitė}
\usepackage{float}
\usepackage{rotating}
\usepackage[pdftex,bookmarks=TRUE]{hyperref}
\usepackage{rotating}
\usepackage{amsmath}
\usepackage{amssymb}
\usepackage{theorem}
\usepackage{calc}
\usepackage{graphicx}
\usepackage{lmodern}
\usepackage{bm}
\usepackage{epsfig}

\newcommand{\R}{R}

\usepackage[noae]{Sweave}
\begin{document}
\Sconcordance{concordance:models_exmpl.tex:models_exmpl.Rnw:%
1 23 1 1 0 4 1 1 27 12 1 1 3 31 0 1 2 1 6 29 0 1 2 4 1 1 6 25 0 1 2 1 6 26 0 2 %
2 1 1 1 15 31 0 1 2 1 6 29 0 2 2 1 1 1 3 9 0 1 2 1 3 8 0 1 2 1 1}


\maketitle
\section{Klausimas}
Trumpai surašysiu modelių pavyzdžius. Galbūt iš to bus kas aiškaus. Dudaitė nagrinėjo tik labai paprasto tipo modelius:
\[ \left\{
  \begin{array}{l l}
    Y_{ij} = \beta_{0j}+r_{ij}; \\
    \beta_{0j} = \gamma_{00} + \gamma_{01}\times X_{j}+u_{0j};
  \end{array} \right.\]
Aš pabandžiau sudėti visus (Dudaitės naudotus) mokyklos kintamuosius bei įtraukti SES (socioekonominė mokinio situacija) kintamąjį. Tokie kintamieji kaip SES ir socialiai remtinų mokinių dalis mokykloje susiję. Tokio modelio nei R, nei HLM7 negalėjo įvertinti. Tad socialiai remtinų mokinių dalies neįtraukiau. Apačioje pateiktas modelio summary:
\[ \left\{
  \begin{array}{l l}
    (mat.rast)_{ij} = \beta_{0j}+\beta_{1j}\times SES_{ij}+r_{ij}; \\
    \beta_{0j} = \gamma_{00} + \gamma_{01}\times (mok.dydis)_{j}+ \gamma_{02}\times (mok.vietove)_{j}+ \gamma_{03}\times (mok.tipas)_{j}+u_{0j};\\
    \beta_{1j} = \gamma_{10}+u_{1j}
  \end{array} \right.\]
\begin{Schunk}
\begin{Soutput}
Linear mixed model fit by REML ['merModLmerTest']
Formula: BSMMAT ~ 1 + MVIETOVE + MDYDIS + MTYPE + SES + (1 + SES | IDSCHOOL) 

REML criterion at convergence: 40987.19 

Random effects:
 Groups   Name        Variance Std.Dev. Corr
 IDSCHOOL (Intercept)  646.74  25.431       
          SES           25.06   5.006   0.01
 Residual             4673.39  68.362       
Number of obs: 3619, groups: IDSCHOOL, 125

Fixed effects:
            Estimate Std. Error t value Pr(>|t|)    
(Intercept) 464.0059    10.2229   45.39   <2e-16 ***
MVIETOVE      3.2809     3.9176    0.84   0.4044    
MDYDIS        0.0163     0.0085    1.92   0.0571 .  
MTYPE         9.3456     4.5221    2.07   0.0415 *  
SES          20.2671     1.4766   13.73   <2e-16 ***
---
Signif. codes:  0 ‘***’ 0.001 ‘**’ 0.01 ‘*’ 0.05 ‘.’ 0.1 ‘ ’ 1

Correlation of Fixed Effects:
         (Intr) MVIETO MDYDIS MTYPE 
MVIETOVE -0.434                     
MDYDIS    0.006 -0.585              
MTYPE    -0.663 -0.022 -0.238       
SES       0.159 -0.127 -0.028 -0.029
\end{Soutput}
\end{Schunk}
Deja, tokie kintamieji kaip mokyklos vietovė ir dydis pasirodė koreliuoti, todėl pasilikau tik mokyklos dydį. Apačioje pateiktas modelio summary:
\begin{Schunk}
\begin{Soutput}
Linear mixed model fit by REML ['merModLmerTest']
Formula: BSMMAT ~ 1 + MDYDIS + MTYPE + SES + (1 + SES | IDSCHOOL) 

REML criterion at convergence: 40992.36 

Random effects:
 Groups   Name        Variance Std.Dev. Corr 
 IDSCHOOL (Intercept)  645.26  25.402        
          SES           26.65   5.162   -0.01
 Residual             4672.20  68.354        
Number of obs: 3619, groups: IDSCHOOL, 125

Fixed effects:
            Estimate Std. Error t value Pr(>|t|)    
(Intercept) 467.5762     9.2054   50.79   <2e-16 ***
MDYDIS        0.0207     0.0069    3.00   0.0034 ** 
MTYPE         9.4064     4.5177    2.08   0.0400 *  
SES          20.3748     1.4704   13.86   <2e-16 ***
---
Signif. codes:  0 ‘***’ 0.001 ‘**’ 0.01 ‘*’ 0.05 ‘.’ 0.1 ‘ ’ 1

Correlation of Fixed Effects:
       (Intr) MDYDIS MTYPE 
MDYDIS -0.339              
MTYPE  -0.746 -0.309       
SES     0.116 -0.128 -0.031
\end{Soutput}
\end{Schunk}
Šis modelis atrodo visai neblogai.

Dar turiu paminėti, kad p-values čia yra gautos aproksimuojant laisvės laipsnius. Be to, pati rašiau multiple imputations suvedimą į vieną modelį pagal HLM7 dokumentaciją ir straipsnius. Tai nežinau, kiek šie rezultatai patikimi.

Jei palikti tik SES:
\begin{Schunk}
\begin{Soutput}
Linear mixed model fit by REML ['merModLmerTest']
Formula: BSMMAT ~ 1 + SES + (1 + SES | IDSCHOOL) 

REML criterion at convergence: 48655.79 

Random effects:
 Groups   Name        Variance Std.Dev. Corr
 IDSCHOOL (Intercept)  684.51  26.163       
          SES           14.54   3.813   1.00
 Residual             4740.03  68.848       
Number of obs: 4292, groups: IDSCHOOL, 143

Fixed effects:
            Estimate Std. Error t value Pr(>|t|)    
(Intercept)  503.595      2.544  197.96   <2e-16 ***
SES           21.511      1.273   16.89   <2e-16 ***
---
Signif. codes:  0 ‘***’ 0.001 ‘**’ 0.01 ‘*’ 0.05 ‘.’ 0.1 ‘ ’ 1

Correlation of Fixed Effects:
    (Intr)
SES 0.047 
\end{Soutput}
\end{Schunk}
Jei tik mokyklos kintamieji:
\begin{Schunk}
\begin{Soutput}
Linear mixed model fit by REML ['merModLmerTest']
Formula: BSMMAT ~ 1 + MDYDIS + MTYPE + (1 | IDSCHOOL) 

REML criterion at convergence: 48215.25 

Random effects:
 Groups   Name        Variance Std.Dev.
 IDSCHOOL (Intercept)  918.8   30.31   
 Residual             4794.5   69.24   
Number of obs: 4245, groups: IDSCHOOL, 125

Fixed effects:
            Estimate Std. Error t value Pr(>|t|)    
(Intercept) 448.8457     9.8150   45.73   <2e-16 ***
MDYDIS        0.0325     0.0075    4.36   <2e-16 ***
MTYPE        13.5755     5.0189    2.70   0.0079 ** 
---
Signif. codes:  0 ‘***’ 0.001 ‘**’ 0.01 ‘*’ 0.05 ‘.’ 0.1 ‘ ’ 1

Correlation of Fixed Effects:
       (Intr) MDYDIS
MDYDIS -0.289       
MTYPE  -0.753 -0.344
\end{Soutput}
\end{Schunk}
Iš tikrųjų nematau daugiau kitų priežasčių dėl negalėjimo sudaryti vieną modelį, kaip tik koreliuoti kintamieji.
\section{Klausimas}
Jei modelius sudaryti tik Vilniui:
\begin{Schunk}
\begin{Soutput}
Linear mixed model fit by REML ['merModLmerTest']
Formula: BSMMAT ~ 1 + MDYDIS + MTYPE + SES + MSOCR2 + (1 + SES | IDSCHOOL) 

REML criterion at convergence: 3183.392 

Random effects:
 Groups   Name        Variance Std.Dev. Corr 
 IDSCHOOL (Intercept)  448.9   21.19         
          SES          234.5   15.31    -0.45
 Residual             3877.0   62.27         
Number of obs: 290, groups: IDSCHOOL, 8

Fixed effects:
            Estimate Std. Error t value Pr(>|t|)  
(Intercept) 470.6883    66.1007   7.121   0.0109 *
MDYDIS        0.0057     0.0196   0.292   0.7889  
MTYPE       -36.7502    26.6464  -1.379   0.3290  
SES          22.7176     7.3792   3.079   0.0260 *
MSOCR2       36.7521    11.1584   3.294   0.0442 *
---
Signif. codes:  0 ‘***’ 0.001 ‘**’ 0.01 ‘*’ 0.05 ‘.’ 0.1 ‘ ’ 1

Correlation of Fixed Effects:
       (Intr) MDYDIS MTYPE  SES   
MDYDIS -0.699                     
MTYPE  -0.793  0.508              
SES    -0.096  0.024  0.052       
MSOCR2 -0.006 -0.267 -0.513 -0.095
\end{Soutput}
\end{Schunk}
Arba:
\begin{Schunk}
\begin{Soutput}
Linear mixed model fit by REML ['merModLmerTest']
Formula: BSMMAT ~ 1 + MDYDIS + MTYPE + SES + (1 + SES | IDSCHOOL) 

REML criterion at convergence: 3587.478 

Random effects:
 Groups   Name        Variance Std.Dev. Corr 
 IDSCHOOL (Intercept)  775.0   27.84         
          SES          264.3   16.26    -0.04
 Residual             3880.8   62.30         
Number of obs: 325, groups: IDSCHOOL, 9

Fixed effects:
            Estimate Std. Error t value Pr(>|t|)   
(Intercept) 478.5399    81.0156   5.907   0.0029 **
MDYDIS        0.0239     0.0287   0.835   0.4441   
MTYPE         5.6217    26.4251   0.213   0.8409   
SES          25.5991     7.2555   3.528   0.0118 * 
---
Signif. codes:  0 ‘***’ 0.001 ‘**’ 0.01 ‘*’ 0.05 ‘.’ 0.1 ‘ ’ 1

Correlation of Fixed Effects:
       (Intr) MDYDIS MTYPE 
MDYDIS -0.730              
MTYPE  -0.895  0.373       
SES     0.026 -0.051 -0.053
\end{Soutput}
\end{Schunk}
Jei imamas vien Vilnius, tai rezultatai kažkiek skiriasi. Buvę reikšmingi kintamieji, tampa nebereikšmingi.
\section{Klausimas}
Dėl kintamųjų koreliacijos. Panašu, kad mokyklos kintamieji kažkiek koreliuoti. Didžiausia korialiacija tarp mokyklos vietovės ir mokyklos dydžio. Bendrai:
\begin{Schunk}
\begin{Soutput}
          MVIETOVE    MDYDIS     MTYPE       SES    MSOCR2
MVIETOVE 1.0000000 0.5854595 0.1617192 0.2638656 0.4514337
MDYDIS   0.5854595 1.0000000 0.1945266 0.1818465 0.4125298
MTYPE    0.1617192 0.1945266 1.0000000 0.2076584 0.2983819
SES      0.2638656 0.1818465 0.2076584 1.0000000 0.2479930
MSOCR2   0.4514337 0.4125298 0.2983819 0.2479930 1.0000000
\end{Soutput}
\end{Schunk}
Vilniuje:
\begin{Schunk}
\begin{Soutput}
            MDYDIS      MTYPE        SES     MSOCR2
MDYDIS  1.00000000 -0.3828944 0.08104535 -0.0294540
MTYPE  -0.38289441  1.0000000 0.13455502  0.3490056
SES     0.08104535  0.1345550 1.00000000  0.1786558
MSOCR2 -0.02945400  0.3490056 0.17865575  1.0000000
\end{Soutput}
\end{Schunk}
Imant tik Vilnių gaunam netgi priešngus kintąmųjų ryšius.
\end{document}
