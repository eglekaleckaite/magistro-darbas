\documentclass[a4paper]{article}
\usepackage{ucs}
\usepackage[utf8x]{inputenc}
\usepackage[L7x]{fontenc}
\usepackage[T1]{fontenc}
\usepackage[lithuanian]{babel}
\title{HLM modeliai TIMSS duomenims\\Nulinis modelis įvertintas keliais būdais}
\author{Eglė Kaleckaitė}
\usepackage{float}
\usepackage{rotating}
\usepackage[pdftex,bookmarks=TRUE]{hyperref}
\usepackage{rotating}
\usepackage{amsmath}
\usepackage{amssymb}
\usepackage{theorem}
\usepackage{calc}
\usepackage{graphicx}
\usepackage{lmodern}
\usepackage{bm}
\usepackage{epsfig}

\newcommand{\R}{R}

\usepackage{Sweave}
\begin{document}
\Sconcordance{concordance:null.models.bootstrap.tex:null.models.bootstrap.Rnw:%
1 23 1 1 0 4 1 1 44 8 1 1 8 17 0 1 2}


\maketitle
\section{Besąlyginis (nulinis) modelis}
Pradžioje pagalvojau, kad reikia su paprasčiausiu modeliu pažaisti:
\[ \left\{
  \begin{array}{l l}
    Y_{ijk} = \pi_{0jk}+r_{ijk}; \\
    \beta_{0k} = \gamma_{000} + u_{0k};
  \end{array} \right.\]
Įretinau su REML, tuomet su saviranka n iš n REML, dar m iš n REML ir galiausiai n iš n atsižvelgiant į hierarchinę struktūra. R'e nebuvo implementuota 3 sluoksnių saviranka, tai padariau klasėms ir mokykloms atskirai. $m = 1000$ ir $n = 4566$.
\section{Rezultatai}


\begin{table}[ht]
\centering
\begin{tabular}{cccccc}
  \hline
 & REML & nnREML & mnREML & nnIDclassREML & HLM7 su svoriais\\ 
  \hline
Intercept & 507.16 & 507.12 & 508.90 & 507.03& 499.25\\ 
  tauBeta00 & 1615.13 & 1877.78 & 1863.90 & 1877.64 & 1653.71\\ 
  sigma & 4201 & 3961 & 3958 & 3961 & 4298.23\\ 
   \hline
\end{tabular}
\end{table}

\end{document}
